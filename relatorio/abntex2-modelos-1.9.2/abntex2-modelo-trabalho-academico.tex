%% abtex2-modelo-trabalho-academico.tex, v-1.9.2 laurocesar
%% Copyright 2012-2014 by abnTeX2 group at http://abntex2.googlecode.com/ 
%%
%% This work may be distributed and/or modified under the
%% conditions of the LaTeX Project Public License, either version 1.3
%% of this license or (at your option) any later version.
%% The latest version of this license is in
%%   http://www.latex-project.org/lppl.txt
%% and version 1.3 or later is part of all distributions of LaTeX
%% version 2005/12/01 or later.
%%
%% This work has the LPPL maintenance status `maintained'.
%% 
%% The Current Maintainer of this work is the abnTeX2 team, led
%% by Lauro César Araujo. Further information are available on 
%% http://abntex2.googlecode.com/
%%
%% This work consists of the files abntex2-modelo-trabalho-academico.tex,
%% abntex2-modelo-include-comandos and abntex2-modelo-references.bib
%%

% ------------------------------------------------------------------------
% ------------------------------------------------------------------------
% abnTeX2: Modelo de Trabalho Academico (tese de doutorado, dissertacao de
% mestrado e trabalhos monograficos em geral) em conformidade com 
% ABNT NBR 14724:2011: Informacao e documentacao - Trabalhos academicos -
% Apresentacao
% ------------------------------------------------------------------------
% ------------------------------------------------------------------------

\documentclass[
	% -- opções da classe memoir --
	12pt,				% tamanho da fonte
	openright,			% capítulos começam em pág ímpar (insere página vazia caso preciso)
	twoside,			% para impressão em verso e anverso. Oposto a oneside
	a4paper,			% tamanho do papel. 
	% -- opções da classe abntex2 --
	%chapter=TITLE,		% títulos de capítulos convertidos em letras maiúsculas
	%section=TITLE,		% títulos de seções convertidos em letras maiúsculas
	%subsection=TITLE,	% títulos de subseções convertidos em letras maiúsculas
	%subsubsection=TITLE,% títulos de subsubseções convertidos em letras maiúsculas
	% -- opções do pacote babel --
	%english,			% idioma adicional para hifenização
	brazil,
	french,				% idioma adicional para hifenização
	%spanish,			% idioma adicional para hifenização
	%brazil				% o último idioma é o principal do documento
	english
	]{abntex2}

% ---
% Pacotes básicos 
% ---
\usepackage{lmodern}			% Usa a fonte Latin Modern			
\usepackage[T1]{fontenc}		% Selecao de codigos de fonte.
\usepackage[utf8]{inputenc}		% Codificacao do documento (conversão automática dos acentos)
\usepackage{lastpage}			% Usado pela Ficha catalográfica
\usepackage{indentfirst}		% Indenta o primeiro parágrafo de cada seção.
\usepackage{color}				% Controle das cores
\usepackage{graphicx}			% Inclusão de gráficos
\usepackage{microtype} 			% para melhorias de justificação
% ---
		
% ---
% Pacotes adicionais, usados apenas no âmbito do Modelo Canônico do abnteX2
% ---
\usepackage{lipsum}				% para geração de dummy text
% ---

% ---
% Pacotes de citações
% ---
\usepackage[brazilian,hyperpageref]{backref}	 % Paginas com as citações na bibl
\usepackage[alf]{abntex2cite}	% Citações padrão ABNT

% --- 
% CONFIGURAÇÕES DE PACOTES
% --- 

% ---
% Configurações do pacote backref
% Usado sem a opção hyperpageref de backref
\renewcommand{\backrefpagesname}{Citado na(s) página(s):~}
% Texto padrão antes do número das páginas
\renewcommand{\backref}{}
% Define os textos da citação
\renewcommand*{\backrefalt}[4]{
	\ifcase #1 %
		Nenhuma citação no texto.%
	\or
		Citado na página #2.%
	\else
		Citado #1 vezes nas páginas #2.%
	\fi}%
% ---

% ---
% Informações de dados para CAPA e FOLHA DE ROSTO
% ---
\titulo{Desenvolvimento de Multiplexador MPEG2 \\ para o Sistema Brasileiro de TV Digital}
\autor{Lucas Pereira Endres}
\local{Brasil}
\data{2014}
\orientador{Altamiro Amadeu Susin}
\coorientador{Sylvie Kerouedan}
\instituicao{%
  Universidade Federal do Rio Grande do Sul
  \par
  Escola de Engenharia
  \par
  Departamento de Engenharia Elétrica}
\tipotrabalho{Trabalho de Conclusão de Curso (Graduação)}
% O preambulo deve conter o tipo do trabalho, o objetivo, 
% o nome da instituição e a área de concentração 
\preambulo{ Trabalho de conclusão de curso apresentado para obtenção do diploma de Engenheiro Eletricista
da Universidade Federal do Rio Grande do Sul.}
% ---


% ---
% Configurações de aparência do PDF final

% alterando o aspecto da cor azul
\definecolor{blue}{RGB}{41,5,195}

% informações do PDF
\makeatletter
\hypersetup{
     	%pagebackref=true,
		pdftitle={\@title}, 
		pdfauthor={\@author},
    	pdfsubject={\imprimirpreambulo},
	    pdfcreator={LaTeX with abnTeX2},
		pdfkeywords={Multiplexação}{MPEG2}{ISDB-T}{Transport Stream}{trabalho acadêmico}, 
		colorlinks=true,       		% false: boxed links; true: colored links
    	linkcolor=blue,          	% color of internal links
    	citecolor=blue,        		% color of links to bibliography
    	filecolor=magenta,      		% color of file links
		urlcolor=blue,
		bookmarksdepth=4
}
\makeatother
% --- 

% --- 
% Espaçamentos entre linhas e parágrafos 
% --- 

% O tamanho do parágrafo é dado por:
\setlength{\parindent}{1.3cm}

% Controle do espaçamento entre um parágrafo e outro:
\setlength{\parskip}{0.2cm}  % tente também \onelineskip

% ---
% compila o indice
% ---
\makeindex
% ---

% ----
% Início do documento
% ----
\begin{document}

% Retira espaço extra obsoleto entre as frases.
\frenchspacing 

% ----------------------------------------------------------
% ELEMENTOS PRÉ-TEXTUAIS
% ----------------------------------------------------------
% \pretextual

% ---
% Capa
% ---
\imprimircapa
% ---

% ---
% Folha de rosto
% (o * indica que haverá a ficha bibliográfica)
% ---
\imprimirfolhaderosto*
% ---

% ---
% Inserir a ficha bibliografica
% ---

% Isto é um exemplo de Ficha Catalográfica, ou ``Dados internacionais de
% catalogação-na-publicação''. Você pode utilizar este modelo como referência. 
% Porém, provavelmente a biblioteca da sua universidade lhe fornecerá um PDF
% com a ficha catalográfica definitiva após a defesa do trabalho. Quando estiver
% com o documento, salve-o como PDF no diretório do seu projeto e substitua todo
% o conteúdo de implementação deste arquivo pelo comando abaixo:
%
% \begin{fichacatalografica}
%     \includepdf{fig_ficha_catalografica.pdf}
% \end{fichacatalografica}
\begin{fichacatalografica}
	\vspace*{\fill}					% Posição vertical
	\hrule							% Linha horizontal
	\begin{center}					% Minipage Centralizado
	\begin{minipage}[c]{12.5cm}		% Largura
	
	\imprimirautor
	
	\hspace{0.5cm} \imprimirtitulo  / \imprimirautor. --
	\imprimirlocal, \imprimirdata-
	
	\hspace{0.5cm} \pageref{LastPage} p. : il. (algumas color.) ; 30 cm.\\
	
	\hspace{0.5cm} \imprimirorientadorRotulo~\imprimirorientador\\
	
	\hspace{0.5cm}
	\parbox[t]{\textwidth}{\imprimirtipotrabalho~--~\imprimirinstituicao,
	\imprimirdata.}\\
	
	\hspace{0.5cm}
		1. Palavra-chave1.
		2. Palavra-chave2.
		I. Orientador.
		II. Universidade xxx.
		III. Faculdade de xxx.
		IV. Título\\ 			
	
	\hspace{8.75cm} CDU 02:141:005.7\\
	
	\end{minipage}
	\end{center}
	\hrule
\end{fichacatalografica}
% ---

% ---
% Inserir errata
% ---
%\begin{errata}
%Elemento opcional da \citeonline[4.2.1.2]{NBR14724:2011}. Exemplo:
%
%\vspace{\onelineskip}
%
%FERRIGNO, C. R. A. \textbf{Tratamento de neoplasias ósseas apendiculares com
%reimplantação de enxerto ósseo autólogo autoclavado associado ao plasma
%rico em plaquetas}: estudo crítico na cirurgia de preservação de membro em
%cães. 2011. 128 f. Tese (Livre-Docência) - Faculdade de Medicina Veterinária e
%Zootecnia, Universidade de São Paulo, São Paulo, 2011.
%
%\begin{table}[htb]
%\center
%\footnotesize
%\begin{tabular}{|p{1.4cm}|p{1cm}|p{3cm}|p{3cm}|}
%  \hline
%   \textbf{Folha} & \textbf{Linha}  & \textbf{Onde se lê}  & \textbf{Leia-se}  \\
%    \hline
%    1 & 10 & auto-conclavo & autoconclavo\\
%   \hline
%\end{tabular}
%\end{table}
%
%\end{errata}
% ---

% ---
% Inserir folha de aprovação
% ---

% Isto é um exemplo de Folha de aprovação, elemento obrigatório da NBR
% 14724/2011 (seção 4.2.1.3). Você pode utilizar este modelo até a aprovação
% do trabalho. Após isso, substitua todo o conteúdo deste arquivo por uma
% imagem da página assinada pela banca com o comando abaixo:
%
% \includepdf{folhadeaprovacao_final.pdf}
%
\begin{folhadeaprovacao}

  \begin{center}
    {\ABNTEXchapterfont\large\imprimirautor}

    \vspace*{\fill}\vspace*{\fill}
    \begin{center}
      \ABNTEXchapterfont\bfseries\Large\imprimirtitulo
    \end{center}
    \vspace*{\fill}
    
    \hspace{.45\textwidth}
    \begin{minipage}{.5\textwidth}
        \imprimirpreambulo
    \end{minipage}%
    \vspace*{\fill}
   \end{center}
        
   Trabalho aprovado. \imprimirlocal, 24 de novembro de 2012:

   \assinatura{\textbf{\imprimirorientador} \\ Orientador} 
   \assinatura{\textbf{Sylvie Kerouedan} \\ Convidado Telecom Bretagne}
   %\assinatura{\textbf{Fulano de Tal} \\ Convidado 2}
   %\assinatura{\textbf{Professor} \\ Convidado 3}
   %\assinatura{\textbf{Professor} \\ Convidado 4}
      
   \begin{center}
    \vspace*{0.5cm}
    {\large\imprimirlocal}
    \par
    {\large\imprimirdata}
    \vspace*{1cm}
  \end{center}
  
\end{folhadeaprovacao}
% ---

% ---
% Dedicatória
% ---
\begin{dedicatoria}
   \vspace*{\fill}
   \centering
   \noindent
   \textit{ Dedicatória ...} \vspace*{\fill}
\end{dedicatoria}
% ---

% ---
% Agradecimentos
% ---
\begin{agradecimentos}
%Os agradecimentos principais são direcionados à Gerald Weber, Miguel Frasson,
%Leslie H. Watter, Bruno Parente Lima, Flávio de Vasconcellos Corrêa, Otavio Real
%Salvador, Renato Machnievscz\footnote{Os nomes dos integrantes do primeiro
%projeto abn\TeX\ foram extraídos de
%\url{http://codigolivre.org.br/projects/abntex/}} e todos aqueles que
%contribuíram para que a produção de trabalhos acadêmicos conforme
%as normas ABNT com \LaTeX\ fosse possível.
%
%Agradecimentos especiais são direcionados ao Centro de Pesquisa em Arquitetura
%da Informação\footnote{\url{http://www.cpai.unb.br/}} da Universidade de
%Brasília (CPAI), ao grupo de usuários
%\emph{latex-br}\footnote{\url{http://groups.google.com/group/latex-br}} e aos
%novos voluntários do grupo
%\emph{\abnTeX}\footnote{\url{http://groups.google.com/group/abntex2} e
%\url{http://abntex2.googlecode.com/}}~que contribuíram e que ainda
%contribuirão para a evolução do \abnTeX.

\end{agradecimentos}
% ---

% ---
% Epígrafe
% ---
%\begin{epigrafe}
%    \vspace*{\fill}
%	\begin{flushright}
%		\textit{``Não vos amoldeis às estruturas deste mundo, \\
%		mas transformai-vos pela renovação da mente, \\
%		a fim de distinguir qual é a vontade de Deus: \\
%		o que é bom, o que Lhe é agradável, o que é perfeito.\\
%		(Bíblia Sagrada, Romanos 12, 2)}
%	\end{flushright}
%\end{epigrafe}
% ---

% ---
% RESUMOS
% ---

% resumo em português
\setlength{\absparsep}{18pt} % ajusta o espaçamento dos parágrafos do resumo
\begin{resumo}[Resumo]
 \begin{otherlanguage*}{brazil}
 
Segundo a \citeonline[3.1-3.2]{NBR6028:2003}, o resumo deve ressaltar o objetivo, o método, os resultados e as conclusões do documento. A ordem e a extensão destes itens dependem do tipo de resumo (informativo ou indicativo) e do tratamento que cada item recebe no documento original. O resumo deve ser precedido da referência do documento, com exceção do resumo inserido no
 próprio documento. (\ldots) As palavras-chave devem figurar logo abaixo do resumo, antecedidas da expressão Palavras-chave:, separadas entre si por ponto e finalizadas também por ponto.

 \textbf{Palavras-chaves}: Multiplexagem. MPEG2. Transport Stream. SBTVD.
 \end{otherlanguage*}
\end{resumo}

% resumo em inglês
\begin{resumo}[Abstract]

   This is the english abstract.

   \vspace{\onelineskip}
 
   \noindent 
   \textbf{Key-words}: Multiplexing. MPEG2. Transport Stream. SBTVD.
 
\end{resumo}

% resumo em francês 
\begin{resumo}[Résumé]
 \begin{otherlanguage*}{french}
 
 Les études pour l'mplantation de la télévision numérique au Brésil ont défini que le système de transmission serait dérivé du japonais, toujours avec des péculiarités. Un des buts des organismes publics en choisissant le standard japonais était de permettre un développement plus significatif de téchnologie nacionale. Les standards américain(ATSC) et européen(DVB) étant déjà bien définis, il ne resterait au Brésil qu'à acheter des dispositifs prêts à l'emploi. Les principales modifications du système japonais sont que nous utilisons un codage MPEG4(H.264) pour la vidéo et les japonais utilisent MPEG2; aussi pour la audio, nous utilisons le HE-AACv2, différement du standard japonais.

Ainsi, le choix pour la téchnologie japonaise modifié a permis aux centres de recherche brésiliens de aider au développement de la téchnologie. Dans ce cadre, il a été confié au Laboratório de Processamento de Sinais (Laboratoire de Traitement du Signal) de mon université la tâche de créer un décodeur du signal transmis: l'entrée est un signal de radio-fréquence d'une chaîne et la sortie est la vidéo et l'audio sur un écran et haut-parleurs.

À fin de contrôller toute la chaîne de transmission, le laboratoire dispose d'un transmisseur de télévision. Ainsi, on peut transmettre sur une chaîne vide un flux vidéo avec des caractéristiques bien définies de codage(codage intra/inter, taille de macroblocs) et tester le décodeur sur une configuration particulière.

Le tâche est, finalement, développer un outil que fasse l'encapsulation des flux (Elementary Streams du MPEG2) vidéo, audio et des donées en un seul flux(Transport Stream du MPEG2). Pour ce faire, on dispose des normes ISO 13818-1 (MPEG2,Systems), ARIB STD-B10(japonaise) et ABNT NBR15603-1(norme brésilienne basée sur la norme japonaise). L'idée est de developper l'outil en langage C, vu que dans ce moment il n'est pas nécessaire que l'encapsulation soit faite en temps réel. En entrée on reçoit des ES vidéo et audio en fichiers binaires et en sortie on rend un fichier binaire avec le TS.

   \textbf{Mots-clés}: Multiplexage. MPEG2. Transport Stream. SBTVD
 \end{otherlanguage*}
\end{resumo}

%% ---
% inserir lista de ilustrações
% ---
\pdfbookmark[0]{\listfigurename}{lof}
\listoffigures*
\cleardoublepage
% ---

% ---
% inserir lista de tabelas
% ---
\pdfbookmark[0]{\listtablename}{lot}
\listoftables*
\cleardoublepage
% ---

% ---
% inserir lista de abreviaturas e siglas
% ---
\begin{siglas}
  \item[ABNT] Associação Brasileira de Normas Técnicas
  \item[ISO] International Standards Organization
  \item[ARIB]
  \item[MPEG] Motion Pictures Experts Group ???
  \item[EBC] Empresa Brasileira de Comunicação
  \item[]
  \item[]
  \item[]
\end{siglas}
% ---

% ---
% inserir lista de símbolos
% ---
%\begin{simbolos}
%  \item[$ \Gamma $] Letra grega Gama
%  \item[$ \Lambda $] Lambda
%  \item[$ \zeta $] Letra grega minúscula zeta
%  \item[$ \in $] Pertence
%\end{simbolos}
% ---

% ---
% inserir o sumario
% ---
\pdfbookmark[0]{\contentsname}{toc}
\tableofcontents*
\cleardoublepage
% ---



% ----------------------------------------------------------
% ELEMENTOS TEXTUAIS
% ----------------------------------------------------------
\textual

% ----------------------------------------------------------
% Introdução (exemplo de capítulo sem numeração, mas presente no Sumário)
% ----------------------------------------------------------
\chapter[Introduction]{Introduction}
%\addcontentsline{toc}{chapter}{Introdução}
% ----------------------------------------------------------

%Historicamente, a televisão está presente nas casas da maioria dos cidadãos brasileiros e é a principal fonte de
%entretenimento e informação para a população. Nos últimos 10 anos, os novos meios de comunicação digital, tais
%como o computador e o telefone celular, estão sendo adotados pela população de diferentes classes sociais. Um
%dado recente \cite{pnad2011} afirma que, em 2011, 69\% da população brasileira dispunha de uma linha de 
%telefone celular. Embora seja significativa, a participação deste meio ainda é muito inferior à da televisão,
%cuja área de cobertura de sinal atinge 100\% do território por via satelital \cite{starone}, e próximo de
%70\% da população por via terrestre. A televisão é, portanto, o principal canal de comunicação disponível ao 
%grande público no Brasil.

%Ainda que tenha cobertura incomparável às demais tecnologias, a televisão é, a grosso modo, um meio de comunicação
%unidirecional. Não é viável, no sistema de transmissão analógica, haver interatividade dos
%telespectadores com o conteúdo apresentado pela geradora. Se comparada à \textit{Internet}, por exemplo, a televisão
%está em desvantagem nesse aspecto. É possível, no entanto, dotar a programação de interatividade se houver um meio de
%retornar dados para a geradora. Com a interatividade, ao usuário podem ser apresentadas opções de programação e, 
%a partir de sua escolha, o televisor exibe o conteúdo. Assim, se poderia aliar a interatividade da \textit{Internet}
%à abrangência da televisão e com isso promover a inclusão social de regiões remotas, sem acesso à infraestrutura 
%de dados em alta velocidade presente nas grandes metrópoles. Essa tecnologia é inviável com a infraestrutura atual
%da televisão analógica: é impossível enviar através de um único canal analógico mais de um conteúdo de vídeo ou
%áudio simultaneamente.
%
%Como solução a essa dificuldade, desenvolveu-se o sistema de transmissão digital descrito pela norma ISO/IEC 13818, comercialmente conhecido pela sigla que dá nome ao comitê formado para redigí-la, o MPEG2. A norma descreve um padrão de codificação e transmissão de vídeo, áudio e dados e que permite, dentre outras funções, transmitir conteúdos audiovisuais independentes simultaneamente, permitindo assim que o usuário do sistema escolha, dentre as informações enviadas pela geradora, qual ele deseja receber.
%
%Desde 1994, empresas privadas e poder público financiaram pesquisas e testes técnicos para comparar o desempenho de três sistemas de televisão digital que na época eram sabidamente eficientes em seus países de origem: o ATSC, desenvolvido nos Estados Unidos; o DVB, desenvolvido pelos países europeus; e o ISDB, desenvolvido no Japão. Os três sistemas têm diferenças no que tange a codificação do vídeo e do áudio, mas o ISDB e o DVB, por exemplo, compartilham a infraestrutura de transporte de dados do padrão MPEG2, mas diferem nos esquemas de modulação do sinal. Após as avaliações, foi concluído que o sistema com a provável melhor performance no território brasileiro seria baseado no ISDB, japonês, com modificações.
%
%As diferenças entre o padrão ISDB original e o adotado no Brasil referem-se principalmente à codificação de vídeo e à plataforma de interatividade. Com o intuito de promover o desenvolvimento da indústria nacional, o governo determinou a adoção de uma plataforma de interatividade de código aberto, desenvolvida majoritariamente com tecnologia da PUC-RJ, o Ginga. Através desta ferramenta, é possível, por exemplo, levar informações de utilidade pública à população de baixa renda, sem acesso
%à \textit{Internet}, como demonstrou a Caixa Econômica Federal em 2010\cite{caixa}. Contudo, a similaridade do sistema de transmissão brasileiro com o padrão internacional MPEG2 e a não obrigatoriedade da utilização de interfaces interativas até 2015 levou ao desinteresse das geradoras no desenvolvimento da tecnologia,
%de modo que atualmente muito pouco se investe para a criação de equipamentos interativos para a televisão digital em terrítório brasileiro.

Historically, television is present in the homes of most Brazilian citizens and is the main source of entertainment and information to the population. Over the past 10 years, new digital media such as computer and cell phones are being adopted by the population of different social classes. A recent data \cite{pnad2011} says that in 2011, 69 \% of the population had a mobile phone line. Although significant, the participation of this medium is still far below the television, whose coverage area reaches 100 \% of the territory via satellite \cite{StarOne}, and around 70 \% of the population by land \cite{forum_sbtvd}. Television is therefore the main channel of communication available to the general public in Brazil.

Although it has unparalleled coverage to other technologies, television is, roughly speaking , a mean of unidirectional communication. It is not possible, using the analog transmission system , to hand to the viewer interactivity with the content submitted by the broadcaster. Compared to \textit{Internet}, for example , television is at disadvantage in this respect. It is possible, however, to provide interactive programming if there is a way to return data to the broadcaster. With interactivity, it can be presented to the user program options, and from the user's choice, the TV displays the chosen content . Thus , one could combine the interactivity of \textit{Internet} to television coverage and thereby promote the social inclusion of remote areas without access to infrastructure of high-speed data links, available in the larger cities . This technology is not feasible with the current analog television infrastructure : it is impossible to ship via a single analog channel more than one video and audio stream simultaneously.

As a solution to this difficulty, ISO/IEC developed the digital transmission system described by ISO/IEC13818 standard, commercially known by the acronym which names the committee formed to draft it, MPEG2. The specification describes standards of encoding and transmitting video, audio and side data. One key feature for television applications is that it is possible to broadcast multiple audiovisual programs simultaneously, thus allowing the user to select which information he wants to receive among several services sent by one single broadcaster, in the same physical channel.

Since 1994 , private companies and government funded research and technical tests to compare the performance of three digital television systems that were known to be efficient in their countries of origin at that time: ATSC \cite{ATSC}, developed in the United States; DVB-T \cite{DVB}, developed by a consortium of companies for use in European countries; and ISDB , developed in Japan by ARIB\cite{ARIB}. The three systems have similarities and differences regarding the encoding of video and audio: for example, both the DVB and ISDB use the transport infrastructure of MPEG2 standard, but differ in the modulation schemes. After the evaluations , it was determined that the system with the best performance for Brazilian territory would be based on the Japanese ISDB with some modifications, such as pointed in \cite{SBTVD_diff_ISDB}.

The differences between the original ISDB and the modified standard used in Brazil mainly relate to video encoding and interactivity platform. In order to promote the development of national industry , the government decided to adopt an open source interactive platform, developed mainly with Ginga, a technology developed mainly by PUC-RJ\cite{PUCRJ}. Through this tool the broadcaster can, for example, deliver useful information to the low-income population without access to the \textit{Internet}, as an application sugested by one Brazilian Bank in 2010 \cite{caixa}. Another application is a project recently developed by the brazilian public comunication company (EBC), called "Brasil 4D" \cite{consultas}. The users can make appointments with doctors or schedule meetings to solve social security issues, or yet check job offers in real time, all this through the TV remote control. However , the similarity of the Brazilian system with the international standard and the fact that interactive interfaces will not be mandatory until 2015 led to the lack of interest of many broadcasters in the development of applications, so that currently very little is invested in the creation of interactive digital television in Brazilian territory.

A digital television system is composed basically by the group of equipments presented in Picture \ref{fig:diagrama_blocos_tvd}. At the beggining of the signal flow, there are the video and audio capturing elements, such as cameras and microphones, and the raw signal may be analog or digital depending on the capture technology used. Once captured, the video and audio are compressed and coded through corresponding encoders. The output of the encoders are standardized bitstreams called Elementary Streams. Once the elementary streams leave the encoders, they enter the multiplexer, where all the streams are packetized and embedded into one single stream, which is the main objective of the MPEG2 systems layer. What follows is one key process for the system robustness: the stream receives a wrapping of error correction code to resist to the noisy, multipath channel between the broadcaster and the receiver. Finally, a digital modulation is applied and the modulated stream is sent to the antenna.
 
 \begin{figure}
\centering
\caption{Block diagram showing digital television basic signal flow.}
\includegraphics[width=1\linewidth]{figuras/diagrama_blocos_tvd.png}
\label{fig:diagrama_blocos_tvd}
\end{figure}
 
The main difference between analog and digital TV broadcasting systems, at the modulation level, is that in analog TV the multiplexing is frequential, i.e., the video signal is sent in a frequency band within the channel bandwidth and the audio in another frequency band. In digital technology, this multiplexing happens in time domain: the multiplexer output has constant bitrate, and for fractions of second only one packet from one of the media sources is sent to the transmitter. In the other side of the channel, the receiver is conceived in such a way that after passing the error correction layer, a \textit{demultiplexer} takes the compound stream and splits it back into elemental bitstreams. They are then sent to the decoders and to audiovisual output devices. It would be certainly disturbing to the viewer to watch delayed video or audio outputs, so one challenge of this multiplexing scheme is to synchronize the reproduction of all the streams. To ensure this, presentation time stamps are added periodically into the streams so they can be reproduced in sync.

%O sistema de transmissão de televisão digital funciona, basicamente, coordenando a entrada de fluxos elementares de vídeo, áudio e dados no fluxo de transmissão. Na televisão analógica, a multiplexação é frequencial, ou seja, o sinal de vídeo é enviado em um intervalo de frequências dentro do canal e o sinal de áudio em outro intervalo. Na tecnologia digital, essa multiplexação é feita no tempo: por frações de segundo, apenas um dos sinais é enviado no canal e o receptor armazena os sinais independentemente para depois fazer a decodificação. No momento da reprodução dos sinais, é preciso que haja sincronismo nos fluxos de vídeo e áudio.

The Presitential Act number 8061\cite{decreto8061}, from july 29, 2013, establishes the chronogram of analog brodcasts shutdown. Until december 31, 2018, all analog transmitters shall be deactivated and the channels shall be vacated. The grants will then be returned to the government, that plans to use the 700MHz band to the 4G mobile telephone service, LTE.

%O Decreto presidencial número 8061, de 29 de julho de 2013, estabelece o cronograma de desligamento dos sistemas de transmissão de televisão analógica. Até 31 de dezembro de 2018, todos os transmissores analógicos devem ser desativados e os canais de radiofrequência devolvidos à união... COLOCAR REFERENCIA! 

The first task of this project is to analyse the brazilian digital television standard and identify the similarities and differences between the original ISO MPEG2 systems layer and the one adopted in Brazil. Once the main differences are established, the next proposed task is to develop a tool that receives as input one or more video and audio files and may provide as an output a binary file with the TS packets, as well as side information, such as language tags and program guides.

The main challenges to accomplish in the project are provide a tool that has four key features: output a stream with an acceptable audiovisual synchronism performance, encode multiple input stream formats into the standard recomendations, create a stream with multiple embedded services and ensure a constant bitrate in the output. In the first few weeks, no freeware solution had been found to meet these requirements, so it was proposed that the tool should be developed from scratch. After extra research efforts, some existing open source applications were found and fortunately they already met some of the requirements, but all were developed respecting the pure MPEG2 standard and naturally didn't include some of the brazilian system requirements. It was then proposed that the project target was to deliver no longer a solution from scratch, bu rather something based on one of the existing tools, adapting it to comprise the brazilian standard requirements. A discussion on the analysed tools is presented in the following chapters.

%% FIM DA INTRODUÇÃO ESCRITA PARA O TCC

%Este documento e seu código-fonte são exemplos de referência de uso da classe
%\textsf{abntex2} e do pacote \textsf{abntex2cite}. O documento 
%exemplifica a elaboração de trabalho acadêmico (tese, dissertação e outros do
%gênero) produzido conforme a ABNT NBR 14724:2011 \emph{Informação e documentação
%- Trabalhos acadêmicos - Apresentação}.
%
%A expressão ``Modelo Canônico'' é utilizada para indicar que \abnTeX\ não é
%modelo específico de nenhuma universidade ou instituição, mas que implementa tão
%somente os requisitos das normas da ABNT. Uma lista completa das normas
%observadas pelo \abnTeX\ é apresentada em \citeonline{abntex2classe}.
%
%Sinta-se convidado a participar do projeto \abnTeX! Acesse o site do projeto em
%\url{http://abntex2.googlecode.com/}. Também fique livre para conhecer,
%estudar, alterar e redistribuir o trabalho do \abnTeX, desde que os arquivos
%modificados tenham seus nomes alterados e que os créditos sejam dados aos
%autores originais, nos termos da ``The \LaTeX\ Project Public
%License''\footnote{\url{http://www.latex-project.org/lppl.txt}}.
%
%Encorajamos que sejam realizadas customizações específicas deste exemplo para
%universidades e outras instituições --- como capas, folha de aprovação, etc.
%Porém, recomendamos que ao invés de se alterar diretamente os arquivos do
%\abnTeX, distribua-se arquivos com as respectivas customizações.
%Isso permite que futuras versões do \abnTeX~não se tornem automaticamente
%incompatíveis com as customizações promovidas. Consulte
%\citeonline{abntex2-wiki-como-customizar} par mais informações.
%
%Este documento deve ser utilizado como complemento dos manuais do \abnTeX\ 
%\cite{abntex2classe,abntex2cite,abntex2cite-alf} e da classe \textsf{memoir}
%\cite{memoir}. 
%
%Esperamos, sinceramente, que o \abnTeX\ aprimore a qualidade do trabalho que
%você produzirá, de modo que o principal esforço seja concentrado no principal:
%na contribuição científica.
%
%Equipe \abnTeX 
%
%Lauro César Araujo


% ----------------------------------------------------------
% PARTE
% ----------------------------------------------------------
%\part{Preparação da pesquisa}
% ----------------------------------------------------------

% ---
% Capitulo com exemplos de comandos inseridos de arquivo externo 
% ---
%\include{abntex2-modelo-include-comandos}
% ---

% ----------------------------------------------------------
% PARTE
% ----------------------------------------------------------
%\part{Referenciais teóricos}
% ----------------------------------------------------------

% ---
% Capitulo de revisão de literatura
% ---

\chapter{Audiovisual codecs used in SBTVD}

The codecs chosen to used in SBTVD are H.264 for video and AAC for the audio. This overview is useful to understand the synchronization methods adopted in the systems layer.

\section{H.264}

The H.264 video coding standard is also known as the MPEG-4 Part 10 standard, and was published as the evolution of the existing video coding standards (ITU-T H.261 / MPEG1, ITU-T H.262 / MPEG2). The main applications for this standard, as described by the publisher itself, are video-conferencing, digital storage media, television broadcasting, Internet streaming and Internet communication. It was developed to provide a solution that has higher compression rates than the former standards, and yet improved subjective quality. To the television broadcast market, it suits perfectly the transmission of high definition video. Technically, the encoder takes the raw video frames and performs sequentially prediction, transform and entropy encoding processes to produce a compressed bitstream. The codec features are organized in profiles and levels, the \textit{high} profile suits the HDTV video transmission characteristics on the Full Segment transmission, while the \textit{baseline} profile fits the needs for mobile, 1-Segment broadcasts.

In the first step, prediction, two types of redundancy in pixel values are exploited: the spatial and the temporal. To do so, the encoder splits one frame into square or rectangular blocks of pixels, called macroblocks, and predicts the values of pixels in the current macroblock based on information from previously analysed macroblocks either in the current frame or in previously coded frames. Then, instead of transmitting the pixel values, is transmits the difference between the original and the predicted frame (which should have less information if the encoder is efficient) along with pointers to the macroblocks that must be used to recreate that region. These pointers are called motion vectors. There are three different types of prediction schemes, that lead to different kinds of frames:

\begin{itemize}
\item{an I-frame is constructed by predicting pixel values only from macroblocks within the same frame, this is the intra-frame prediction;}
\item{a P-frame is composed of predictions from the current source frame and also of the previously coded I-frame, this is the inter-frame prediction;}
\item{a B-frame is made by predicting the output frame from the current input, the earlier inputs or yet the later input frames, and is known as the bidirectional inter-frame prediction.}
\end{itemize}

The decrease in amount of information can be noticed in Figure \ref{fig:motion_compensation}, adapted from \cite{richardson}. The first frame is marked as (a) and coded with intra prediction. The following frame is (b), which is coded as a P-frame. When generic motion compensation coding is applied, the residual frame is the one in (c). Knowing that these frames have differential pixel values \footnote{Differential pixel values means that a black pixel is the negative threshold, a white pixel is the positive threshold and gray is zero.}, it can be noticed that most of the pixels have a value close to zero in (c), which means a residual with very little information. The number of bytes needed to code one I frame is therefore much higher than the necessary to code a P frame, since there is no motion compensation in the first one.

\begin{figure}
\centering
\caption{Visualization of motion compensation encoder working.}
\includegraphics[width=1\linewidth]{figuras/motion_compensation.png}
\\Source: \cite{richardson}.
\label{fig:motion_compensation}
\end{figure}

Although there is a substantial decrease in amount of information with inter-frame prediction, the video quality may degenerate catastrophically if too many P and B frames are used between two I frames. This leads to a compromise of quality and bitrate and the usual frame arrangement is the one shown in Figure \ref{fig:IBBPBBP}, that places at most two B frames between one I and one P, or between two P frames, and at most two P frames between consecutive I frames. This arrangement is called a Group of Pictures, or GOP as referenced in MPEG4 standard.

\begin{figure}
\centering
\caption{.}
\includegraphics[width=1\linewidth]{figuras/IBBPBBP.png}
\\Source: The Author.
\label{fig:IBBPBBP}
\end{figure}

The next encoder step is to transform the residual frame from prediction with an integer approximation of the Discrete Cosine Transform, or DCT, so that the pixel values become a set of frequency domain coefficients. These coefficients are then quantized, which reduce their precision and make many of them to be equal to zero. Finally, the coefficients are reordered to align most of the zero-valued ones consecutively, and an entropy encoder reduces the redundancy of the sequences of repetitions, hence reducing the amount of data. The output of the H.264 encoder is formed of the quantized coefficients and the motion vectors arranged in a structured bitstream, that constitutes the video Elementary Stream. Further information can be found in \cite{vcodex} or \cite{richardson}.

\section{AAC}

The audio codec chosen to incorporate the SBTVD, developed by Fraunhofer IIS along with AT\&T, Dolby and Sony, is called Advanced Audio Coding, or AAC. Since the beggining of the dewvelopment in 1994, the successive evolutions in the codec gave place to profiles that assemble the main features. Three different profiles are the mostly used in digital television broadcast: AAC-LC, HE-AAC, HE-AACv2. According to the main developer(\cite{fraunhofer}), even the simplest profile, AAC-LC (LC for low complexity) provides \textit{an audio whose quality even for expert listeners, so-called “golden ears”, is indistinguishable from the original}.

The codec supports encoding of ancillary data along with the audio samples, and all with typically 192Kbps for broadcast quality stereo streams in AAC-LC profile. If the HE-AAC profile is chosen, the Spectral Band Replication(SBR) feature is applied and for the same audio quality in the output, a 30\% bitrate reduction is observed. In HE-AAC the lower part of the audio spectrum is sampled at half the frequency of the remaining signal using AAC-LC, and the high frequency band is coded with SBR. SBR exploits the relationship of upper and lower sides of the spectrum to produce a parametric recreation of the upper band, thus reducing the total bitrate. HE-AAC achieves typical 48 to 64 kbps for stereo streams.

The chosen profile to integrate the SBTVD is HE-AACv2. Additionally to the already described features, it includes the Parametric Stereo (PS) tool. Rather than encoding two separate channels without any relationship between them, PS creates a monaural downmix of the two stereo samples and a set of parametric coefficients. The mono stream is then encoded with HE-AAC and sent along with the coefficients to recreate stereo sound in the receiver side. Finally, with both PS and SBR, the bitrate is reduced even more, to up to 24 Kbps for stereo audio. \autoref{fig:he_aac_v2} presents the encoding and decoding processes for HE-AACv2 profile. The audio stream enters the encoder and reaches first the Parametric Stereo step. The set of coefficients are extracted, injected as ancillary data and the residual enters both the SBR encoder and the downsampler. The SBR coefficients are calculated and added to the ancillary data, and the final residual is coded with AAC-LC.

\begin{figure}
\centering
\caption{Block diagram of HE-AACv2 profile encoding and decoding steps.}
\includegraphics[width=1\linewidth]{figuras/he_aac_v2.png}
\\Source and Copyright: \cite{fraunhofer}.
\label{fig:he_aac_v2}
\end{figure}

\chapter{ISO/IEC 13818-1 Standard}
% ---
The main objective of the ISO/IEC13818-1 standard is to describe the conversion of multiple bitstreams into one single stream carrying wrapped video and audio data, as well as additional information responsible to unwrap the streams. Similarly to other ISO/IEC standards, the text only specifies the decoding / demultiplexing side of the transmission chain, leaving the architectures of encoding / multiplexing stages to the manufacturers decision. The standard defines two different architectures that should be applied in oposing situations, both of them based on the transmission of packets of data. The first is the Program Stream, defined to be used in error free situations, such as storage in digital discs: the streams of this type are composed of long packets of variable length, the Packetized Elementary Streams (PES) packets. To demonstrate the variable length characteristic, consider the video elementary stream as an example: the common size for one PES packet is one video frame, since there are considerable size differences in I-frames and B-frames, the PES packet size may vary. The other architecture, suitable for faulty, error prone transmission channels, is the Transport Stream. The stream packet has fixed length and is much smaller than the PES packets, which makes it easier to error correction codes to ensure the transmission through the faulty channels. In fact, one PES packet size can be hundreds of times bigger than one TS packet, because the multiplexer breaks the PES packets to fit them inside the TS packets. Since the second architecture is much more suitable to broadcasting situations, the following text will be focused only on it.

% ---
\section{Transport Stream}
% ---
The main structure of the MPEG2 Transport Stream is its constant size packet. The stream is made of many packets concatenated. The Picture \ref{fig:TS_iso13818} shows the basic elements that are present in one packet, the value under each field is its length in bits. From the picture, it can be seen that a packet is roughly formed with 188 bytes of data. The mandatory header has 4 bytes, and goes from the \textit{sync byte} to the \textit{continuity counter}. The packet payload might also contain the \textit{Adaptation Field}, a sort of header extension, with additional data that helps to decode and present the streams more efficiently.  Apart from the 4 mandatory bytes for the header, the other 184 bytes may be filled in three different ways: with the Adaptation Field, with PES data or with information tables. The Adaptation Field presence is optional, and it is commonly used to fill with stuffing bits the last TS packets from one PES packet.

\begin{figure}
\centering
\caption{Formation scheme of a TS packet.}
\includegraphics[width=1\linewidth]{figuras/TS_iso13818.png}
\\Source: ISO/IEC13818-1.
\label{fig:TS_iso13818}
\end{figure}

%Na presente seção é descrito o Transport Stream do MPEG2 com suas tabelas PSI. A seguir, na figura \ref{fig:TS_iso13818}, retirada da norma ISO13818 tal qual, é apresentado um esquema da construção de um pacote do TS. Da figura, vê-se que um pacote de TS é formado por 188 bytes de dados. O cabeçalho obrigatório tem 4 bytes, e vai até o campo \textit{continuity counter}. Os dados do pacote podem ainda incluir o \textit{Adaptation Field}, espécie de extensão do cabeçalho, e com informções adicionais para a decodificação do TS. Os fluxos elementares de mídia, os ES, são divididos em pacotes de comprimento variável e formam um PES. Um pacote de TS deve conter o cabeçalho obrigatório com 4 bytes, e a ele devem suceder 184 bytes de dados. Esses bytes restantes podem ser preenchidos de diferentes maneiras:

%\begin{itemize}
%\item{ somente com o \textit{Adaptation Field};}
%\item{ ou com este e mais dados de pacotes PES , que são os fluxos elementares de mídia separados em pacotes;}
%\item{ou ainda com sessões de tabelas de dados complementares aos fluxos, fundamentais para a parametrização do decodificador.}
%\end{itemize}

The ISO13818 standard recommends the TS bitrate to be constant, according to \ref{ISO_constant_bitrate}. However, to increase the codecs performance, their output  is necessarily variable, and therefore stuffing bits must be added to the transport streams whenever there is no useful information to be transmitted. To better understand this, take the video encoder as an example: consider that the frame rate is constant and equal to 29.97 frames per second. Since the frames are coded according to the H.264 standard with motion compensation and prediction techniques, the I-frames have necessarily more information than P-frames or even more than B-frames. Therefore, during the transmission of an I-frame, the amount of useful information in the output stream is higher than if a B-frame is being processed. To illustrate this variation of rates, a graphic representation of the transitions is presented in Figure \ref{fig:rate_transitions}. It is up to the muxer to handle this variable to constant transformation, as will be discussed in the following sections.

\begin{figure}
\centering
\caption{Variation of rates among the transport stream formation.}
\includegraphics[width=1\linewidth]{figuras/rate_transitions.png}
\\Source: The author.
\label{fig:rate_transitions}
\end{figure}

% ---

\section{Synchronism}

%O sincronismo é fundamental para o funcionamento de um sistema de transmissão de televisão, pois é preciso garantir a reprodução do vídeo e do áudio simultaneamente e com a mesma referência temporal. Para possibilitar tal sincronismo, um complexo sistema de sincronismo existe no padrão. O conceito
fundamental é a transmissão de um sinal de clock codificado nos pacotes do TS, o PCR, explicado a seguir. Esse sinal é decodificado e alimenta um sistema PLL no decodificador, para então ser utilizado nos fluxos de áudio e vídeo. Os fluxos elementares, por sua vez, transportam referências temporais ao PCR nos campos PTS e DTS, explicados a seguir, e assim são reproduzidos com base em uma mesma referência temporal.

Timing is crucial to the functioning of a television broadcasting system, because it is necessary to guarantee the reproduction of video and multiple audio streams simultaneously. To enable such a timing, a complex system is defined in the standard. The key concept is to define a System Time Clock, sample this clock and transmit it encoded in the TS packets. The STC uses the NLL (Numerically Locked Loop) technology, equivalent to PLL, but digital. In the decoder side, a PLL module recreates the original STC which then is used in the audio and video streams sync process. Elementary streams, in their turn, carry temporal references to the STC and are played based on the same time reference as the encoder. The field that carries the STC samples is the PCR, detailed below. Elementary Streams carry timing information in the PTS and DTS fields.

\begin{figure}[!hb]
\centering
\caption{PCR Sampling block diagram.}
\includegraphics[width=1*\linewidth]{figuras/pcr_sampling.png}
\\Source and Copyright: The Author.
\label{fig:pcr_sampling}
\end{figure}

\subsection{PCR}
O valor do PCR é uma espécie de amostragem do clock do encoder de vídeo ou áudio em um momento específico da geração da unidade fundamental, seja um quadro de áudio ou vídeo, se o sistema trabalhar em tempo real. Como no caso desde multiplexador não é, só é preciso definir a frequência, que é definida pela norma em 27MHz, e calcular os valores sequenciais do PCR baseando-se na taxa de bits por segundo de cada um dos sinais de vídeo, ou áudio.

\begin{equation}
$$ PCR = F*ES_{BR}  $$
\end{equation}

O PCR é enviado em  um campo do cabeçalho do TS, e está dentro do adaptation field. Embora seja opcional enviá-lo,
nota-se pela leitura que é possível enviar o PCR em poucas repetições. Não é necessário enviá-lo junto
com todos os pacotes de TS. Pode-se enviar um pacote apenas com o PCR, utilizando um PCR\_ID, para resincronizar o
PLL do receptor a cada N segundos, N a definir.


\subsection{PTS / DTS}

Os campos Decoding Time Stamp \(DTS\) e Presentation Time Stamp \(PTS\) são outros dois referenciais de tempo,
derivados do PCR, e que servem a informar ao decodificador o instante de tempo em que devem ser decodificadas
e exibidas, respectivamente, as Access Units de vídeo e áudio. Uma Presentation Unit \(PU\) é definida pela
norma como sendo um quadro de áudio ou de vídeo, e uma Access Unit \(AU\) é a representação codificada 
da Presentation Unit.

A norma define claramente qual instante de tempo deve ser considerado para gerar o PTS ou o DTS:
\quotation{No caso do áudio, se um PTS está presente num cabeçalho de PES, este deve fazer referência à
primeira AU que comece no pacote. Uma AU de áudio começa em um pacote se o primeiro byte da AU de áudio
estiver presente no pacote. No caso do vídeo, se um PTS está presente no cabeçalho do pacote PES, este
deve se referir à AU que cujo primeiro \textit{start\_code} está neste pacote.}

% ---

An example of an hypothetical decoder initial operation is shown in Figure \ref{fig:funcionamento_inicial_decoder}. Utility of retransmission of PSI tables from time to time 

\begin{figure}[!hb]
\centering
\caption{Hypothetical decoder initial operation.}
\includegraphics[width=1*\linewidth]{figuras/funcionamento_inicial_decoder.png}
\\Source and Copyright: \cite{nhk}.
\label{fig:funcionamento_inicial_decoder}
\end{figure}


\section{PSI}

PSI stands for Program Specific Information and serves to organize transmission, timing, decoding and presentation information of audio, video and data streams. The PSI set is organized in tables, each with a specific function in the standard. The tables that serve to indicate the decoder where are the streams of video and audio, called PAT (Program Association Table) and PMT (Table Mapping Program) are mandatory. There is another set of tables that are not mandatory according to ISO13818, but ABNT NBR15603 defines as mandatory.


%O PSI é o conjunto de informações especificas ao programa, ou \textit{Program Specific information}, em inglês, e serve para organizar as informações de transmissão, sincronismo, decodificação e apresentação dos fluxos de áudio, vídeo e dados. O PSI é organizado em forma de tabelas, cada uma com uma função específica no padrão. As tabelas que servem a indicar ao decodificador onde estão no TS os fluxos de vídeo e áudio, chamadas de PAT( Tabela de associação de programas) e PMT (Tabela de Mapeamento de Programas), são obrigatórias, Há ainda um conjunto grande de tabelas que servem a outros fins, e que não são obrigatórias segundo a ISO13818, mas que a ABNT NBR15603 define como obrigatórias.

The ISO standard defines that the PSI tables must be sent in sections and that the sections can be segmented to fit the TS packets. Sections can not be longer than 1024 bytes, although in practice the number of bytes for the mandatory tables is well below that . The syntax of the tables is standardized. A field at the beginning of the table (table\_ID) defines what is its type (PAT, PMT, etc..) and section number field serves as a counter if the table is divided into several sections.

TODO TODO The field version Umos fields that follow are customized version, section number, program number

%A norma ISO define que as tabelas do PSI devem ser enviadas em seções e as seções podem ser segmentadas para caberem nos pacotes de TS. As seções não podem ter mais de 1024 bytes, embora na prática o número de bytes para uma tabela obrigatória fique bem abaixo. A sintaxe das tabelas é padronizada. Um campo no início da tabela (table\_ID) define qual o seu tipo (PAT,PMT, etc.) e o campo section number serve como um contador caso a tabela seja dividida em diversas seções. O campo version umos campos que seguem são personalizados version, section number, program number.

\subsection{PAT}

The Program Association Table (PAT) indicates the PID of the corresponding Program Map Table (PMT) for each service in the multiplexer. Technically, it links a \textit{program number}, which is the numeric representation of a service, to the PID of that program's PMT, the \textit{program map PID}. Also, it must contain the information for the Network Information Table (NIT). Figure \ref{fig:TSAnalyser_close_PAT}, that follows, shows the output of the MPEG TS Analyser when a PAT table section is analysed. The field \textit{table id} indicates 0x0, which is the PAT code. The pair \textit{program\_number} 0 and \textit{program\_map\_PID} 16 is unique and its presence indicates that the NIT table is present and is carried in the PID 16. It's unique because, according to the ISO standard, a program number 0 always indicates the NIT PID, and the NIT table shall always use the PID 16. The other pairs indicate that there are two programs in the transport stream, with program numbers 23104 and 23129. Each of these programs have their PMTs allocated in the PIDs 257 and 8136, respectively. In the PMT section, it will be seen that these values are considered. <<< TODO TODO

\begin{figure}[!hb]
\centering
\caption{TS Analyser output showing a PAT Table.}
\includegraphics[width=0.4\linewidth]{figuras/TSAnalyser_close_PAT.png}
\\Source: Reproduction of the MPEG TS Analyser software.
\label{fig:TSAnalyser_close_PAT}
\end{figure}

\subsection{PMT}

The Program Map Table (PMT) indicates the PID of the elementary streams corresponding to its service. Technicaly, it links a \textit{program number} to several \textit{elementary PIDs}. Each elementary stream must have a \textit{type} identification, which designate the nature of the stream, according to ISO13818, and defines roughly whether it is a video, audio or data stream. Table \ref{tab_ISOESTypes}, that follows, is an excerpt of the standardized types.

\begin{table}[!htpd]
\caption{Some Standardized Elementary Stream Types.}
\begin{center}
\begin{tabular}{|c|c|c|}
\hline
Decimal & Hex & Description \\
Value & Value & \\
\hline
6 & 0x06 & ITU-T Rec. H.222 and ISO/IEC 13818-1 (MPEG-2 packetized data)\\
 & & privately defined (ie, DVB subtitles/VBI and AC-3)\\
 \hline
17 & 0x11 & ISO/IEC 14496-3 (MPEG-4 LOAS multi-format framed audio)\\
 & & in a packetized stream \\
 \hline
27 & 0x1B & ITU-T Rec. H.264 and ISO/IEC 14496-10 (lower bit-rate video)\\
 & & in a packetized stream \\
\hline
\end{tabular}
\label{tab_ISOESTypes}
\\Source: \cite{ISOESTypes}
\end{center}
\end{table}

Figure \ref{fig:TSAnalyser_close_PMT}, that follows, shows a snapshot of the analyser output for a PMT table. In the example, the \textit{program number} is 23104, the table ID corresponding to the PMT is 0x02 (as defined by ISO13818-1) and the PCR information is carried in the PID (PCR\_PID) 256d \footnote{The \textit{d} indicates a decimal number.}. Yet, from Picture \ref{fig:TSAnalyser_close_PMT_TS_Header}, it can be seen at the TS Header that the PID carying the PMT table is number 257, as indicated by the PAT table analysed before.

By analyzing both Table \ref{tab_ISOESTypes} and Picture \ref{fig:TSAnalyser_close_PMT}, one may notice that the stream types and PIDs that come after the PCR information match: the ES with PID 273 is H.264 video and the ESs with PIDs 274 and 275 are AAC/LATM audio streams, which means that there are two audio streams. The stream with type 6d and PID 288 is the subtitle / closed captions stream. Later on this chapter the descriptors shown here will be discussed.

\begin{figure}[!hb]
\centering
\caption{TS Analyser output showing a PMT Table.}
\includegraphics[width=0.4\linewidth]{figuras/TSAnalyser_close_PMT.png}
\\Source: Reproduction of the MPEG TS Analyser software.
\label{fig:TSAnalyser_close_PMT}
\end{figure}

\begin{figure}[!hb]
\centering
\caption{TS Analyser output showing the TS Header info for the PMT Table.}
\includegraphics[width=0.4\linewidth]{figuras/TSAnalyser_close_PMT_TS_Header.png}
\\Source: Reproduction of the MPEG TS Analyser software.
\label{fig:TSAnalyser_close_PMT_TS_Header}
\end{figure}

\subsection{NIT}

\chapter{Soluções existentes e limitaçoes}

Uma busca rápida em um motor de pesquisa pelas palavras-chave \textit{MPEG2 TS muxer} retorna uma infinidade de
soluçoes comerciais e gratuitas para a multiplexaçao de arquivos de mídia dos mais diferentes formatos no padrão
MPEG2. As soluçoes comerciais geralmente estão associadas a \textit{hardwares} dedicados, como a apresentada por
\cite{harris} e custam elevadas somas, da ordem de alguns milhares de dólares. Estes equipamentos funcionam em tempo
real, recebendo fluxos de vídeo, áudio e dados em múltiplas interfaces de entrada ASI e entregando na saída um fluxo
multiplexado no padrão MPEG2. Os principais clientes destas soluções são geradoras de televisão, que produzem conteúdo
ao vivo e precisam de baixa latência na codificação do sinal. As soluções gratuitas são \textit{softwares} compatíveis com as plataformas Windows e Linux, na sua maioria. Alguns têm interface gráfica e configuraçoes padrão para auxiliar o seu uso por usuários pouco familiarizados com o padrão. A principal diferença é que o objetivo não é a utilizaçao do sistema em tempo real, de modo que as especificaçoes de processamento e memória dos computadores pessoais geralmente são
suficientes para executar as ferramentas. As interfaces de entrada e saída são comumente arquivos binários que armazenam os fluxos de dados de maneira sequencial.

Para fins de elaboraçao da ferramenta de multiplexação proposta neste trabalho, algumas ferramentas gratuitas foram escolhidas para estudo. Para a escolha destas ferramentas em particular, os critérios foram os seguintes:

\begin{itemize}

\item{as ferramentas deveriam ser de código aberto e com licença freeware, para manter os aspectos legais no projeto;}

\item{as ferramentas deveriam possibilitar a modificaçao do código para adaptar-se o máximo possivel a situaçao atual de cada programa à norma brasileira;}

\item{o desenvolvimento das ferramentas deveria estar ativo, ou seja, não se considerou soluções antigas, que não tenham sofrido atualizaçoes nos últimos 5 anos;}

\item{as ferramentas precisam apresentar compatibilidade nativa com os fluxos elementares padrão da norma brasileira, como as codificações de vídeo H.264 e áudio AAC empacotado por LATM.}

\end{itemize}

As soluções encontradas que atendem a essas condiçoes são apresentadas na sequência, com detalhes das características individuais.

%==> julgou-se inicialmente que ela deveria ser desenvolvida desde o início, mas após a pesquisa 

%Multiplos Serviços(program number)
%Encoder Fluxos Video e Audio(h264, aac-latm)
%Garantia de Sincronismo (calculo do PTS/PCR)
%Criaçao de Fluxo TS com taxa de bits constante


\section{FFMPEG}

Segundo os próprios desenvolvedores, o FFMPEG é uma ferramenta versátil de codificaçao, decodificação, verificaçao e exibiçao de fluxos de vídeo, áudio e legendas. Com um conjunto vasto de bibliotecas de código aberto, permite conversão entre diferentes formatos, taxa de quadros, tamanho de quadro de vídeo, amostragem de áudio, dentre outras funçoes. É a solução mais completa e em mais expressivo desenvolvimento, com atualizações diárias organizadas em um repositório público.

Nativamente, não permite a criaçao de fluxos TS com múltiplos serviços(ou programas). É capaz de entregar na saída vídeos codificados em formato H.264 e áudios AAC-LATM. É possível manter o sincronismo entre os fluxos, quer mantendo as informaçoes presentes nos fluxos originais, quer gerando novas marcações temporais. Se desejado, uma taxa constante de bits pode ser programada e o resultado é confiável.

\section{GPAC}

Ferramenta desenvolvida pela Escola Francesa Télécom ParisTech, apresenta características similares à anterior, mas é menos performante no que compete ao sincronismo e à manutenção de uma taxa constante de bits no arquivo de saída. Suporta a criação de múltiplos serviços, mas apresenta comportamento aleatório no número de pacotes gerados. Observou-se diversas vezes que, para arquivos de entrada com 10 segundos de duraçao e as mesmas configurações da ferramenta, eventualmente eram gerados desde 7 segundos até 15 segundos, o que sugere uma falta de controle do sistema. Consequentemente, quando o arquivo era gerado com tempo diferente do original, a taxa de reprodução dos fluxos ficava alterada e o resultado não era aproveitável para exibição.

\section{TSTools}

Esta ferramenta é bastante simples, ainda não foi suficientemente explorada. Já foi modificada por \cite{Borin} para acrescentar as tabelas da norma brasileira a um arquivo TS existente.

\section{EiTV}

Não é gratuita, mas é a soluçao disponível no laboratório e também apresenta funcionalidades de multiplexaçao. Dispõe de hardware dedicado para a multiplexação e também tem conexão com um transmissor digital.É possível entregar ao equipamento um arquivo em formato TS com pacotes de 188 bytes, mas não com múltiplos serviços.

\section{Available Solutions Choice}

Even though FFMPEG doesn't provide native support for multiple MPEG services, it was chosen as the base solution to the project. It was noticed that the modifications to the source code in order to add multiple service support were not extensive. The working sinchronism feature counted towards this option, too. Even though FFMPEG has a public API that could be used to develop a muxing application from scratch, the main task of this project is already done by the \textit{ffmpeg-formats} util .The missing features are the main target of this project:

\begin{itemize}

\item{add support to multiple service MPEG2 transport streams;}

\item{add the tables which are optional in the ISO/IEC standard but are obligatory in SBTVD;}

\item{create a more user-friendly interface to FFMPEG, simplify the operation of the command-line interface.}

\end{itemize}

\chapter{Architecture of the chosen solution}

FFMPEG is organized in the following structure: there are four executable applications that result from the package compilation: \textit{ffmpeg, ffprobe, ffplay} and \textit{ffserver}. These four applications run based on seven libraries that are shared among the applications, and can be also used outside FFMPEG via an API. The \textit{libavformat} library is responsible for the muxing and demuxing functionalities, and therefore includes a ISO/IEC 13818-1 compatible muxer.

The main MPEG2 muxer structure is \textit{MpegTSWrite}, which has a excerpt presented in \ref{tab_MpegTSWrite}. The structure contains many control variables, as well as the PSI tables PAT and SDT as \textit{MpegTSSection} structs, a dynamic array \textit{services} and a counter \textit{nb\_services} with respect to the services contained in the TS, the output muxing rate \( \textit{mux\_rate} \) and the PIDs for tables and elementary streams \( \textit{service\_id} and \textit{pmt\_start\_pid} \) .

\begin{table}[!htpd]
\label{tab_MpegTSWrite}
\caption{ Excerpt of MpegTSWrite structure.}
\begin{center}
\begin{tabular}{|l|}
\hline
\\
typedef struct MpegTSWrite \{\\
...\\
MpegTSSection pat;\\
MpegTSSection sdt;\\
MpegTSService **services;\\
...\\
int nb\_services;\\
...\\
int mux\_rate;\\
...\\
int service\_id;\\
int pmt\_start\_pid;\\
...\\
 \}MpegTSWrite;\\
 \\
\hline
\end{tabular}
\end{center}
\end{table}

The next relevant structure to the development of this project is MpegTSSection, which is presented in Table \ref{tab_MpegTSSection} . This is the responsible for holding a generic table representation of what is described by ISO/IEC standard, with a PID, a continuity counter \(\textit{cc}\), a pointer to the function that writes in a TS packet the corresponding table and another pointer, \textit{opaque}, that directs to the general AVFormatContext structure. This is quite a general section, the customization for each of the PSI tables is done by the functions that write the sections to the stream, which will be commented further in this text.

\begin{table}[!htpd]
\label{tab_MpegTSSection}
\caption{ Excerpt of MpegTSSection structure.}
\begin{center}
\begin{tabular}{|l|}
\hline
\\
typedef struct MpegTSSection \{\\
	int pid;\\
	int cc;\\
	void\(*write\_packet\)\(struct MpegTSSection *s, const uint8\_t *packet\);\\
	void *opaque;\\
\} MpegTSSection;\\
 \\
\hline
\end{tabular}
\end{center}
\end{table}

What follows is the structure that organizes the information concerning each MPEG service in the transport stream. The \textit{MpegTSService} struct, presented in Table \ref{tab_MpegTSService} holds the following information: a pointer to the corresponding PMT, the service ID, the provider and service names, as character strings, and the PCR PID related to this service.

\begin{table}[!htpd]
\label{tab_MpegTSService}
\caption{ Excerpt of MpegTSService structure.}
\begin{center}
\begin{tabular}{|l|}
\hline
\\
typedef struct MpegTSService \{
MpegTSSection pmt; \\
int sid; \\
char *name;\\
char *provider\_name;\\
int pcr\_pid;\\
...\\
\}MpegTSService;\\
 \\
\hline
\end{tabular}
\end{center}
\end{table}

The last of the selected MpegTS structures is \textit{MpegTSWriteStream}, which is formed by the information describing each of the elementary streams that will eventually be included in the services. The service that contains one stream is specified here via a pointer to a service structure, as one can see from Table \ref{tab_MpegTSWriteStream}. Along with the service, there are also the data specifying the stream PID and the current continuity counter value.

\begin{table}[!htpd]
\label{tab_MpegTSWriteStream}
\caption{ Excerpt of MpegTSWriteStream structure.}
\begin{center}
\begin{tabular}{|l|}
\hline
\\
typedef struct MpegTSWriteStream \{\\
struct MpegTSService *service;\\
int pid;\\
int cc;\\
...\\
\}MpegTSWriteStream;\\
 \\
\hline
\end{tabular}
\end{center}
\end{table}

By knowing the few information about FFMPEG muxer structures presented so far, one can already understand basically how the system works. A simple block diagram in Figure \ref{fig:diagrama_estruturas_ffmpeg} shows the relations between the presented structures, along with an example situation. Consider a digital television transport stream constituted by two services, with service IDs 1 and 2. The first service is formed by one video stream, with ID 101, and two audio streams, with IDs 111 and 112. The second service has one video stream, ID 201, and one audio stream, with ID 211. Each service has its own mapping table: for service 1 the table has a Table ID \(TID\) of 4001, and for service 2 the TID is 4002. Each program map table \(PMT\) links the service number to the stream numbers as defined by the standard. Finally, the program association table \(PAT\) links the available services 1 and 2 with their respective PMTs. To possibilitate the decoder to find these information within a transport stream, the PAT must have TID equal to 0, as can be noticed.

\begin{figure}
\centering
\caption{Block diagram showing the relations between FFMPEG MPEG2 Muxer.}
\includegraphics[width=1\linewidth]{figuras/diagrama_estruturas_ffmpeg.png}
\\Source: the author.
\label{fig:diagrama_estruturas_ffmpeg}
\end{figure}

This paragraph is a free adaptation of FFMPEG documentation for general muxing processes. The main functions related to the muxing process are avformat\_write\_header() for writing the TS stream header, av\_write\_frame() for writing the packets and av\_write\_trailer() for writting the stream end. After setting up the muxer environment, a call to the write\_header() function starts the transport stream output. After that, sucessive calls to write\_frame() and write\_section() fill in the output stream with video and audio streams or table sections, respectively. Finally, the stream is ended by calling write\_trailler().


% ----------------------------------------------------------
% PARTE
% ----------------------------------------------------------
%\part{Resultados}
% ----------------------------------------------------------

% ---
% primeiro capitulo de Resultados
% ---
\chapter{Project implementations}
% ---

% ---
\section{Multiple services implementation}
% ---

The first proposed modification to the original FFMPEG code is add multiple service support to the main program. For that objective, the first task was to find a way to give the muxer new information about how many services should be coded from the command-line interface. Fortunately, the application counts with a \textit{options} structure, which delivers to the muxer a group of pairs \textit{<option,value>}. The option \textit{final\_nb\_services} was created to guide the muxer to create the desired number of services. Through the existent interface, the user may provide multiple input files, with multiple audio/video streams, to form a single service, multiple stream output. By modifying the service assignment process, this implementation allows the user to create multiple services, multiple stream outputs.

When calling FFMPEG, the user maps the input streams which shall be placed in each service by using the \textit{\-map} option. Each input stream is assigned an internal stream number by FFMPEG, and the muxer uses this data, combined to the newly added option, to alternately assign one stream to each service created with a modulus operation of the two information. Considering the example shown in Figure \ref{fig:diagrama_estruturas_ffmpeg}, the mapping should be done by placing first the two video streams, then the first audio stream corresponding to the first service, then the audio stream corresponding to the second service and finally the second audio stream corresponding to the first service.

As can be seen on the code excerpt in Table \ref{tab_service_assignment}, once the number of services parameter is set from the command line, first the muxer loops to create services by calling mpegts\_add\_service() function. Along with the service creation, some fields of the new services are initialized, such as the proper table packets writing function, the global table context, and its continuity counter. When the services are added to the \textit{ts} context, they are appended to the end of a dynamic array of services (\textit{ts->services}). In a second moment, the muxer loops in the streams and assigns alternately one service to each stream. The assignment is done using the \textit{service} pointer field in the MpegTSWriteStream structure, presented before, here initialized as \textit{ts\_st}. The index for the service array is the modulus operation result between the current stream and the total number of services.

TODO !!!

\begin{table}[!htpd]
\label{tab_service_assignment}
\caption{ Excerpt of service assignment algorithm.}
\begin{center}
\begin{tabular}{|l|}
\hline
\\
...\\
for(i = 0;i < ts->final_nb_services; i++) \{\\
service = mpegts\_add\_service(ts, ts->service\_id+i, provider\_name, service\_name);\\
service->pmt.write\_packet = section\_write\_packet;\\
service->pmt.opaque = s;\\
service->pmt.cc = 15;\\
\}\\
...\\
for(i = 0;i < s->nb\_streams; i++) \{\\
...\\
ts\_st->service = ts->services[i \% ts->final\_nb\_services] ;\\
...
\}\\
\\
\hline
\end{tabular}
\end{center}
\end{table}

% ---
%\section{Desempenho em velocidade de multiplexação}
% ---

%Para quantificar o tempo que o software desenvolvido leva para tratar fluxos de vídeo e áudio de tamanhos específicos, utilizou-se o comando \texttt{time}.

%\lipsum[21-22]

%% ---
%% segundo capitulo de Resultados
%% ---
%\chapter{Nam sed tellus sit amet lectus urna ullamcorper tristique interdum
%elementum}
%% ---
%
%% ---
%\section{Pellentesque sit amet pede ac sem eleifend consectetuer}
%% ---
%
%\lipsum[24]

% ----------------------------------------------------------
% Finaliza a parte no bookmark do PDF
% para que se inicie o bookmark na raiz
% e adiciona espaço de parte no Sumário
% ----------------------------------------------------------
\phantompart

% ---
% Conclusão (outro exemplo de capítulo sem numeração e presente no sumário)
% ---
\chapter[Conclusions]{Conclusions}
%\addcontentsline{toc}{chapter}{Conclusão}
% ---

%\lipsum[31-33]
%Aqui serão escritas as conclusões do trabalho, com seriedade e racionalidade.


% ---
% Perspectivas (outro exemplo de capítulo sem numeração e presente no sumário)
% ---
%\chapter*[Perspectivas]{Perspectivas}
%\addcontentsline{toc}{chapter}{Perspectivas}
% ---

%\lipsum[31-33]
%Aqui serão escritas as perspectivas futuras para a continuação do desenvolvimento, com 
%comentários sobre o fato de que ainda haverá tempo para trabalhar no projeto até final
%de agosto.



% ----------------------------------------------------------
% ELEMENTOS PÓS-TEXTUAIS
% ----------------------------------------------------------
\postextual
% ----------------------------------------------------------

% ----------------------------------------------------------
% Referências bibliográficas
% ----------------------------------------------------------
\bibliography{abntex2-modelo-trabalho-academico}

% ----------------------------------------------------------
% Glossário
% ----------------------------------------------------------
%
% Consulte o manual da classe abntex2 para orientações sobre o glossário.
%
%\glossary

% ----------------------------------------------------------
% Apêndices
% ----------------------------------------------------------

% ---
% Inicia os apêndices
% ---
\begin{apendicesenv}

% Imprime uma página indicando o início dos apêndices
\partapendices

% ----------------------------------------------------------
\chapter{Tabelas selecionadas retiradas da norma ISO/IEC 13818-1}
% ----------------------------------------------------------

%\lipsum[50]

% ----------------------------------------------------------
\chapter{Trecho do código implementado em linguagem C}
% ----------------------------------------------------------
%\lipsum[55]

\end{apendicesenv}
% ---


% ----------------------------------------------------------
% Anexos
% ----------------------------------------------------------

% ---
% Inicia os anexos
% ---
\begin{anexosenv}

% Imprime uma página indicando o início dos anexos
\partanexos

% ---
%\chapter{Trechos extraídos da norma ISO/IEC 13818}
% ---
%\lipsum[30]

% ---
%\chapter{Cras non urna sed feugiat cum sociis natoque penatibus et magnis dis
%parturient montes nascetur ridiculus mus}
% ---

%\lipsum[31]

% ---
%\chapter{Fusce facilisis lacinia dui}
% ---

%\lipsum[32]

\end{anexosenv}

%---------------------------------------------------------------------
% INDICE REMISSIVO
%---------------------------------------------------------------------
\phantompart
\printindex
%---------------------------------------------------------------------

\end{document}
