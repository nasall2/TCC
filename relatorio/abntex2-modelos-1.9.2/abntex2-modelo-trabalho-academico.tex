%% abtex2-modelo-trabalho-academico.tex, v-1.9.2 laurocesar
%% Copyright 2012-2014 by abnTeX2 group at http://abntex2.googlecode.com/ 
%%
%% This work may be distributed and/or modified under the
%% conditions of the LaTeX Project Public License, either version 1.3
%% of this license or (at your option) any later version.
%% The latest version of this license is in
%%   http://www.latex-project.org/lppl.txt
%% and version 1.3 or later is part of all distributions of LaTeX
%% version 2005/12/01 or later.
%%
%% This work has the LPPL maintenance status `maintained'.
%% 
%% The Current Maintainer of this work is the abnTeX2 team, led
%% by Lauro César Araujo. Further information are available on 
%% http://abntex2.googlecode.com/
%%
%% This work consists of the files abntex2-modelo-trabalho-academico.tex,
%% abntex2-modelo-include-comandos and abntex2-modelo-references.bib
%%

% ------------------------------------------------------------------------
% ------------------------------------------------------------------------
% abnTeX2: Modelo de Trabalho Academico (tese de doutorado, dissertacao de
% mestrado e trabalhos monograficos em geral) em conformidade com 
% ABNT NBR 14724:2011: Informacao e documentacao - Trabalhos academicos -
% Apresentacao
% ------------------------------------------------------------------------
% ------------------------------------------------------------------------

\documentclass[
	% -- opções da classe memoir --
	12pt,				% tamanho da fonte
	openright,			% capítulos começam em pág ímpar (insere página vazia caso preciso)
	twoside,			% para impressão em verso e anverso. Oposto a oneside
	a4paper,			% tamanho do papel. 
	% -- opções da classe abntex2 --
	%chapter=TITLE,		% títulos de capítulos convertidos em letras maiúsculas
	%section=TITLE,		% títulos de seções convertidos em letras maiúsculas
	%subsection=TITLE,	% títulos de subseções convertidos em letras maiúsculas
	%subsubsection=TITLE,% títulos de subsubseções convertidos em letras maiúsculas
	% -- opções do pacote babel --
	english,			% idioma adicional para hifenização
	french,				% idioma adicional para hifenização
	%spanish,			% idioma adicional para hifenização
	brazil				% o último idioma é o principal do documento
	]{abntex2}

% ---
% Pacotes básicos 
% ---
\usepackage{lmodern}			% Usa a fonte Latin Modern			
\usepackage[T1]{fontenc}		% Selecao de codigos de fonte.
\usepackage[utf8]{inputenc}		% Codificacao do documento (conversão automática dos acentos)
\usepackage{lastpage}			% Usado pela Ficha catalográfica
\usepackage{indentfirst}		% Indenta o primeiro parágrafo de cada seção.
\usepackage{color}				% Controle das cores
\usepackage{graphicx}			% Inclusão de gráficos
\usepackage{microtype} 			% para melhorias de justificação
% ---
		
% ---
% Pacotes adicionais, usados apenas no âmbito do Modelo Canônico do abnteX2
% ---
\usepackage{lipsum}				% para geração de dummy text
% ---

% ---
% Pacotes de citações
% ---
\usepackage[brazilian,hyperpageref]{backref}	 % Paginas com as citações na bibl
\usepackage[alf]{abntex2cite}	% Citações padrão ABNT

% --- 
% CONFIGURAÇÕES DE PACOTES
% --- 

% ---
% Configurações do pacote backref
% Usado sem a opção hyperpageref de backref
\renewcommand{\backrefpagesname}{Citado na(s) página(s):~}
% Texto padrão antes do número das páginas
\renewcommand{\backref}{}
% Define os textos da citação
\renewcommand*{\backrefalt}[4]{
	\ifcase #1 %
		Nenhuma citação no texto.%
	\or
		Citado na página #2.%
	\else
		Citado #1 vezes nas páginas #2.%
	\fi}%
% ---

% ---
% Informações de dados para CAPA e FOLHA DE ROSTO
% ---
\titulo{Desenvolvimento de Multiplexador MPEG2 \\ para o Sistema Brasileiro de TV Digital}
\autor{Lucas Pereira Endres}
\local{Brasil}
\data{2014}
\orientador{Altamiro Amadeu Susin}
\coorientador{Sylvie Kerouedan}
\instituicao{%
  Universidade Federal do Rio Grande do Sul
  \par
  Escola de Engenharia
  \par
  Departamento de Engenharia Elétrica}
\tipotrabalho{Trabalho de Conclusão de Curso (Graduação)}
% O preambulo deve conter o tipo do trabalho, o objetivo, 
% o nome da instituição e a área de concentração 
\preambulo{ Trabalho de conclusão de curso apresentado para obtenção do diploma de Engenheiro Eletricista
da Universidade Federal do Rio Grande do Sul.}
% ---


% ---
% Configurações de aparência do PDF final

% alterando o aspecto da cor azul
\definecolor{blue}{RGB}{41,5,195}

% informações do PDF
\makeatletter
\hypersetup{
     	%pagebackref=true,
		pdftitle={\@title}, 
		pdfauthor={\@author},
    	pdfsubject={\imprimirpreambulo},
	    pdfcreator={LaTeX with abnTeX2},
		pdfkeywords={Multiplexação}{MPEG2}{ISDB-T}{Transport Stream}{trabalho acadêmico}, 
		colorlinks=true,       		% false: boxed links; true: colored links
    	linkcolor=blue,          	% color of internal links
    	citecolor=blue,        		% color of links to bibliography
    	filecolor=magenta,      		% color of file links
		urlcolor=blue,
		bookmarksdepth=4
}
\makeatother
% --- 

% --- 
% Espaçamentos entre linhas e parágrafos 
% --- 

% O tamanho do parágrafo é dado por:
\setlength{\parindent}{1.3cm}

% Controle do espaçamento entre um parágrafo e outro:
\setlength{\parskip}{0.2cm}  % tente também \onelineskip

% ---
% compila o indice
% ---
\makeindex
% ---

% ----
% Início do documento
% ----
\begin{document}

% Retira espaço extra obsoleto entre as frases.
\frenchspacing 

% ----------------------------------------------------------
% ELEMENTOS PRÉ-TEXTUAIS
% ----------------------------------------------------------
% \pretextual

% ---
% Capa
% ---
\imprimircapa
% ---

% ---
% Folha de rosto
% (o * indica que haverá a ficha bibliográfica)
% ---
\imprimirfolhaderosto*
% ---

% ---
% Inserir a ficha bibliografica
% ---

% Isto é um exemplo de Ficha Catalográfica, ou ``Dados internacionais de
% catalogação-na-publicação''. Você pode utilizar este modelo como referência. 
% Porém, provavelmente a biblioteca da sua universidade lhe fornecerá um PDF
% com a ficha catalográfica definitiva após a defesa do trabalho. Quando estiver
% com o documento, salve-o como PDF no diretório do seu projeto e substitua todo
% o conteúdo de implementação deste arquivo pelo comando abaixo:
%
% \begin{fichacatalografica}
%     \includepdf{fig_ficha_catalografica.pdf}
% \end{fichacatalografica}
\begin{fichacatalografica}
	\vspace*{\fill}					% Posição vertical
	\hrule							% Linha horizontal
	\begin{center}					% Minipage Centralizado
	\begin{minipage}[c]{12.5cm}		% Largura
	
	\imprimirautor
	
	\hspace{0.5cm} \imprimirtitulo  / \imprimirautor. --
	\imprimirlocal, \imprimirdata-
	
	\hspace{0.5cm} \pageref{LastPage} p. : il. (algumas color.) ; 30 cm.\\
	
	\hspace{0.5cm} \imprimirorientadorRotulo~\imprimirorientador\\
	
	\hspace{0.5cm}
	\parbox[t]{\textwidth}{\imprimirtipotrabalho~--~\imprimirinstituicao,
	\imprimirdata.}\\
	
	\hspace{0.5cm}
		1. Palavra-chave1.
		2. Palavra-chave2.
		I. Orientador.
		II. Universidade xxx.
		III. Faculdade de xxx.
		IV. Título\\ 			
	
	\hspace{8.75cm} CDU 02:141:005.7\\
	
	\end{minipage}
	\end{center}
	\hrule
\end{fichacatalografica}
% ---

% ---
% Inserir errata
% ---
%\begin{errata}
%Elemento opcional da \citeonline[4.2.1.2]{NBR14724:2011}. Exemplo:
%
%\vspace{\onelineskip}
%
%FERRIGNO, C. R. A. \textbf{Tratamento de neoplasias ósseas apendiculares com
%reimplantação de enxerto ósseo autólogo autoclavado associado ao plasma
%rico em plaquetas}: estudo crítico na cirurgia de preservação de membro em
%cães. 2011. 128 f. Tese (Livre-Docência) - Faculdade de Medicina Veterinária e
%Zootecnia, Universidade de São Paulo, São Paulo, 2011.
%
%\begin{table}[htb]
%\center
%\footnotesize
%\begin{tabular}{|p{1.4cm}|p{1cm}|p{3cm}|p{3cm}|}
%  \hline
%   \textbf{Folha} & \textbf{Linha}  & \textbf{Onde se lê}  & \textbf{Leia-se}  \\
%    \hline
%    1 & 10 & auto-conclavo & autoconclavo\\
%   \hline
%\end{tabular}
%\end{table}
%
%\end{errata}
% ---

% ---
% Inserir folha de aprovação
% ---

% Isto é um exemplo de Folha de aprovação, elemento obrigatório da NBR
% 14724/2011 (seção 4.2.1.3). Você pode utilizar este modelo até a aprovação
% do trabalho. Após isso, substitua todo o conteúdo deste arquivo por uma
% imagem da página assinada pela banca com o comando abaixo:
%
% \includepdf{folhadeaprovacao_final.pdf}
%
%\begin{folhadeaprovacao}
%
%  \begin{center}
%    {\ABNTEXchapterfont\large\imprimirautor}
%
%    \vspace*{\fill}\vspace*{\fill}
%    \begin{center}
%      \ABNTEXchapterfont\bfseries\Large\imprimirtitulo
%    \end{center}
%    \vspace*{\fill}
%    
%    \hspace{.45\textwidth}
%    \begin{minipage}{.5\textwidth}
%        \imprimirpreambulo
%    \end{minipage}%
%    \vspace*{\fill}
%   \end{center}
%        
%   Trabalho aprovado. \imprimirlocal, 24 de novembro de 2012:
%
%   \assinatura{\textbf{\imprimirorientador} \\ Orientador} 
%   \assinatura{\textbf{Sylvie Kerouedan} \\ Convidado Telecom Bretagne}
%   %\assinatura{\textbf{Fulano de Tal} \\ Convidado 2}
%   %\assinatura{\textbf{Professor} \\ Convidado 3}
%   %\assinatura{\textbf{Professor} \\ Convidado 4}
%      
%   \begin{center}
%    \vspace*{0.5cm}
%    {\large\imprimirlocal}
%    \par
%    {\large\imprimirdata}
%    \vspace*{1cm}
%  \end{center}
%  
%\end{folhadeaprovacao}
% ---

% ---
% Dedicatória
% ---
\begin{dedicatoria}
   \vspace*{\fill}
   \centering
   \noindent
   \textit{ Dedicatória ...} \vspace*{\fill}
\end{dedicatoria}
% ---

% ---
% Agradecimentos
% ---
\begin{agradecimentos}
%Os agradecimentos principais são direcionados à Gerald Weber, Miguel Frasson,
%Leslie H. Watter, Bruno Parente Lima, Flávio de Vasconcellos Corrêa, Otavio Real
%Salvador, Renato Machnievscz\footnote{Os nomes dos integrantes do primeiro
%projeto abn\TeX\ foram extraídos de
%\url{http://codigolivre.org.br/projects/abntex/}} e todos aqueles que
%contribuíram para que a produção de trabalhos acadêmicos conforme
%as normas ABNT com \LaTeX\ fosse possível.
%
%Agradecimentos especiais são direcionados ao Centro de Pesquisa em Arquitetura
%da Informação\footnote{\url{http://www.cpai.unb.br/}} da Universidade de
%Brasília (CPAI), ao grupo de usuários
%\emph{latex-br}\footnote{\url{http://groups.google.com/group/latex-br}} e aos
%novos voluntários do grupo
%\emph{\abnTeX}\footnote{\url{http://groups.google.com/group/abntex2} e
%\url{http://abntex2.googlecode.com/}}~que contribuíram e que ainda
%contribuirão para a evolução do \abnTeX.

\end{agradecimentos}
% ---

% ---
% Epígrafe
% ---
%\begin{epigrafe}
%    \vspace*{\fill}
%	\begin{flushright}
%		\textit{``Não vos amoldeis às estruturas deste mundo, \\
%		mas transformai-vos pela renovação da mente, \\
%		a fim de distinguir qual é a vontade de Deus: \\
%		o que é bom, o que Lhe é agradável, o que é perfeito.\\
%		(Bíblia Sagrada, Romanos 12, 2)}
%	\end{flushright}
%\end{epigrafe}
% ---

% ---
% RESUMOS
% ---

% resumo em português
\setlength{\absparsep}{18pt} % ajusta o espaçamento dos parágrafos do resumo
\begin{resumo}
% Segundo a \citeonline[3.1-3.2]{NBR6028:2003}, o resumo deve ressaltar o
% objetivo, o método, os resultados e as conclusões do documento. A ordem e a extensão
% destes itens dependem do tipo de resumo (informativo ou indicativo) e do
% tratamento que cada item recebe no documento original. O resumo deve ser
% precedido da referência do documento, com exceção do resumo inserido no
% próprio documento. (\ldots) As palavras-chave devem figurar logo abaixo do
% resumo, antecedidas da expressão Palavras-chave:, separadas entre si por
% ponto e finalizadas também por ponto.

 \textbf{Palavras-chaves}: Multiplexagem. MPEG2. Transport Stream. SBTVD.
\end{resumo}

% resumo em inglês
\begin{resumo}[Abstract]
 \begin{otherlanguage*}{english}
   This is the english abstract.

   \vspace{\onelineskip}
 
   \noindent 
   \textbf{Key-words}: latex. abntex. text editoration.
 \end{otherlanguage*}
\end{resumo}

% resumo em francês 
\begin{resumo}[Résumé]
 \begin{otherlanguage*}{french}
 
 Les études pour l'mplantation de la télévision numérique au Brésil ont défini que le système de transmission serait dérivé du japonais, toujours avec des péculiarités. Un des buts des organismes publics en choisissant le standard japonais était de permettre un développement plus significatif de téchnologie nacionale. Les standards américain(ATSC) et européen(DVB) étant déjà bien définis, il ne resterait au Brésil qu'à acheter des dispositifs prêts à l'emploi. Les principales modifications du système japonais sont que nous utilisons un codage MPEG4(H.264) pour la vidéo et les japonais utilisent MPEG2; aussi pour la audio, nous utilisons le HE-AACv2, différement du standard japonais.

Ainsi, le choix pour la téchnologie japonaise modifié a permis aux centres de recherche brésiliens de aider au développement de la téchnologie. Dans ce cadre, il a été confié au Laboratório de Processamento de Sinais (Laboratoire de Traitement du Signal) de mon université la tâche de créer un décodeur du signal transmis: l'entrée est un signal de radio-fréquence d'une chaîne et la sortie est la vidéo et l'audio sur un écran et haut-parleurs.

À fin de contrôller toute la chaîne de transmission, le laboratoire dispose d'un transmisseur de télévision. Ainsi, on peut transmettre sur une chaîne vide un flux vidéo avec des caractéristiques bien définies de codage(codage intra/inter, taille de macroblocs) et tester le décodeur sur une configuration particulière.

Le tâche est, finalement, développer un outil que fasse l'encapsulation des flux (Elementary Streams du MPEG2) vidéo, audio et des donées en un seul flux(Transport Stream du MPEG2). Pour ce faire, on dispose des normes ISO 13818-1 (MPEG2,Systems), ARIB STD-B10(japonaise) et ABNT NBR15603-1(norme brésilienne basée sur la norme japonaise). L'idée est de developper l'outil en langage C, vu que dans ce moment il n'est pas nécessaire que l'encapsulation soit faite en temps réel. En entrée on reçoit des ES vidéo et audio en fichiers binaires et en sortie on rend un fichier binaire avec le TS.

   \textbf{Mots-clés}: Multiplexage. MPEG2. ISDB-T. Transport Stream.
 \end{otherlanguage*}
\end{resumo}

%% ---
% inserir lista de ilustrações
% ---
\pdfbookmark[0]{\listfigurename}{lof}
\listoffigures*
\cleardoublepage
% ---

% ---
% inserir lista de tabelas
% ---
%\pdfbookmark[0]{\listtablename}{lot}
%\listoftables*
%\cleardoublepage
% ---

% ---
% inserir lista de abreviaturas e siglas
% ---
\begin{siglas}
  \item[ABNT] Associação Brasileira de Normas Técnicas
  \item[ISO] International Standards Organization
  \item[ARIB]
  \item[MPEG2]
  \item[]
  \item[]
  \item[]
  \item[]
\end{siglas}
% ---

% ---
% inserir lista de símbolos
% ---
%\begin{simbolos}
%  \item[$ \Gamma $] Letra grega Gama
%  \item[$ \Lambda $] Lambda
%  \item[$ \zeta $] Letra grega minúscula zeta
%  \item[$ \in $] Pertence
%\end{simbolos}
% ---

% ---
% inserir o sumario
% ---
\pdfbookmark[0]{\contentsname}{toc}
\tableofcontents*
\cleardoublepage
% ---



% ----------------------------------------------------------
% ELEMENTOS TEXTUAIS
% ----------------------------------------------------------
\textual

% ----------------------------------------------------------
% Introdução (exemplo de capítulo sem numeração, mas presente no Sumário)
% ----------------------------------------------------------
\chapter[Introdução]{Introdução}
%\addcontentsline{toc}{chapter}{Introdução}
% ----------------------------------------------------------

Historicamente, a televisão está presente nas casas da maioria dos cidadãos brasileiros e é a principal fonte de
entretenimento e informação para a população. Nos últimos 10 anos, os novos meios de comunicação digital, tais
como o computador e o telefone celular, estão sendo adotados pela população de diferentes classes sociais. Um
dado recente \cite{pnad2011} afirma que, em 2011, 69\% da população brasileira dispunha de uma linha de 
telefone celular. Embora seja significativa, a participação deste meio ainda é muito inferior à da televisão,
cuja área de cobertura de sinal atinge 100\% do território por via satelital \cite{starone}, e próximo de
70\% da população por via terrestre. A televisão é, portanto, o principal canal de comunicação disponível ao 
grande público no Brasil.

Ainda que tenha cobertura incomparável às demais tecnologias, a televisão é, a grosso modo, um meio de comunicação
unidirecional. Não é viável, no sistema de transmissão analógica, haver interatividade dos
telespectadores com o conteúdo apresentado pela geradora. Se comparada à \textit{Internet}, por exemplo, a televisão
está em desvantagem nesse aspecto. É possível, no entanto, dotar a programação de interatividade se houver um meio de
retornar dados para a geradora. Com a interatividade, ao usuário podem ser apresentadas opções de programação e, 
a partir de sua escolha, o televisor exibe o conteúdo. Assim, se poderia aliar a interatividade da \textit{Internet}
à abrangência da televisão e com isso promover a inclusão social de regiões remotas, sem acesso à infraestrutura 
de dados em alta velocidade presente nas grandes metrópoles. Essa tecnologia é inviável com a infraestrutura atual
da televisão analógica: é impossível enviar através de um único canal analógico mais de um conteúdo de vídeo ou
áudio simultaneamente.

Como solução a essa dificuldade, desenvolveu-se
o sistema de transmissão digital descrito pela norma ISO/IEC 13818, comercialmente conhecido pela sigla que dá nome ao
comitê formado para redigí-la, o MPEG2. A norma descreve um padrão de codificação e transmissão de vídeo, áudio e dados
e que permite, dentre outras funções, transmitir conteúdos audiovisuais independentes simultaneamente, permitindo
assim que o usuário do sistema escolha, dentre as informações enviadas pela geradora, qual ele deseja receber.

Desde 1994, empresas privadas e poder público financiaram pesquisas e testes técnicos para comparar o desempenho de
três sistemas de televisão digital que na época eram sabidamente eficientes em seus países de origem: o ATSC,
desenvolvido nos Estados Unidos; o DVB, desenvolvido pelos países europeus; e o ISDB, desenvolvido no Japão. Os três
sistemas têm diferenças no que tange a codificação do vídeo e do áudio, mas o ISDB e o DVB, por exemplo, compartilham
a infraestrutura de transporte de dados do padrão MPEG2, mas diferem nos esquemas de modulação do sinal. Após as
avaliações, foi concluído que o sistema com a provável melhor performance no território brasileiro seria baseado
no ISDB, japonês, com modificações.

As diferenças entre o padrão ISDB original e o adotado no Brasil referem-se principalmente à codificação de vídeo e à
plataforma de interatividade. Com o intuito de promover o desenvolvimento da indústria nacional, o governo determinou a adoção de uma
plataforma de interatividade de código aberto, desenvolvida majoritariamente com tecnologia da PUC-RJ, o Ginga.
Através desta ferramenta, é possível, por exemplo, levar informações de utilidade pública à população de baixa renda, sem acesso
à \textit{Internet}, como demonstrou a Caixa Econômica Federal em 2010\cite{caixa}.
Contudo, a similaridade do sistema de transmissão brasileiro com o padrão internacional MPEG2 e a não
obrigatoriedade da utilização de interfaces interativas até 2015 levou ao desinteresse das geradoras no desenvolvimento da tecnologia,
de modo que atualmente muito pouco se investe para a criação de equipamentos interativos para a televisão digital em terrítório
brasileiro.

O sistema de transmissão para 

--> explicar a importância da tv digital, que não só melhora a qualidade do sinal, mas também possibilita 
a integração?

 --> Explicar o que é o "sistema de transmissão" !!!
 
 --> previsão do fim da TV analógica em 2017, tendência de mais emissoras migrarem para o Digital...
 
O sistema de transmissão de televisão digital funciona, basicamente, coordenando a entrada de fluxos elementares de
vídeo, áudio e dados no fluxo de transmissão. Na televisão analógica, a multiplexação é frequencial, ou seja, o sinal
de vídeo é enviado em um intervalo de frequências dentro do canal e o sinal de áudio em outro intervalo. Na tecnologia
digital, essa multiplexação é feita no tempo: por frações de segundo, apenas um dos sinais é enviado no canal e o
receptor armazena os sinais independentemente para depois fazer a decodificação. No momento da reprodução dos
sinais, é preciso que haja sincronismo nos fluxos de vídeo e áudio.

O Decreto presidencial número 8061, de 29 de julho de 2013, estabelece o cronograma de desligamento dos sistemas
de transmissão de televisão analógica. Até 31 de dezembro de 2018, todos os transmissores analógicos devem ser desativados
e os canais de radiofrequência devolvidos à união... COLOCAR REFERENCIA! 

http://www.planalto.gov.br/ccivil_03/_Ato2011-2014/2013/Decreto/D8061.htm
DECRETO Nº 8.061, DE 29 DE JULHO DE 2013

%% FIM DA INTRODUÇÃO ESCRITA PARA O TCC

%Este documento e seu código-fonte são exemplos de referência de uso da classe
%\textsf{abntex2} e do pacote \textsf{abntex2cite}. O documento 
%exemplifica a elaboração de trabalho acadêmico (tese, dissertação e outros do
%gênero) produzido conforme a ABNT NBR 14724:2011 \emph{Informação e documentação
%- Trabalhos acadêmicos - Apresentação}.
%
%A expressão ``Modelo Canônico'' é utilizada para indicar que \abnTeX\ não é
%modelo específico de nenhuma universidade ou instituição, mas que implementa tão
%somente os requisitos das normas da ABNT. Uma lista completa das normas
%observadas pelo \abnTeX\ é apresentada em \citeonline{abntex2classe}.
%
%Sinta-se convidado a participar do projeto \abnTeX! Acesse o site do projeto em
%\url{http://abntex2.googlecode.com/}. Também fique livre para conhecer,
%estudar, alterar e redistribuir o trabalho do \abnTeX, desde que os arquivos
%modificados tenham seus nomes alterados e que os créditos sejam dados aos
%autores originais, nos termos da ``The \LaTeX\ Project Public
%License''\footnote{\url{http://www.latex-project.org/lppl.txt}}.
%
%Encorajamos que sejam realizadas customizações específicas deste exemplo para
%universidades e outras instituições --- como capas, folha de aprovação, etc.
%Porém, recomendamos que ao invés de se alterar diretamente os arquivos do
%\abnTeX, distribua-se arquivos com as respectivas customizações.
%Isso permite que futuras versões do \abnTeX~não se tornem automaticamente
%incompatíveis com as customizações promovidas. Consulte
%\citeonline{abntex2-wiki-como-customizar} par mais informações.
%
%Este documento deve ser utilizado como complemento dos manuais do \abnTeX\ 
%\cite{abntex2classe,abntex2cite,abntex2cite-alf} e da classe \textsf{memoir}
%\cite{memoir}. 
%
%Esperamos, sinceramente, que o \abnTeX\ aprimore a qualidade do trabalho que
%você produzirá, de modo que o principal esforço seja concentrado no principal:
%na contribuição científica.
%
%Equipe \abnTeX 
%
%Lauro César Araujo


% ----------------------------------------------------------
% PARTE
% ----------------------------------------------------------
%\part{Preparação da pesquisa}
% ----------------------------------------------------------

% ---
% Capitulo com exemplos de comandos inseridos de arquivo externo 
% ---
%\include{abntex2-modelo-include-comandos}
% ---

% ----------------------------------------------------------
% PARTE
% ----------------------------------------------------------
%\part{Referenciais teóricos}
% ----------------------------------------------------------

% ---
% Capitulo de revisão de literatura
% ---
\chapter{Norma ISO/IEC 13818-1}
% ---

% ---
\section{Transport Stream}
% ---

Aqui será descrito o Transport Stream do MPEG2, com suas tabelas PSI. A seguir, na figura
\ref{fig:TS_iso13818}, retirada da norma ISO13818 tal qual, é apresentado um esquema da construção de
um pacote do TS. Da figura, vê-se que um pacote de TS é formado por 188 bytes de dados. O cabeçalho
obrigatório tem 4 bytes, e vai até o campo \textit{continuity counter}. Os dados do pacote podem ainda
incluir o \textit{Adaptation Field}, espécie de extensão do cabeçalho, e com informções adicionais para
a decodificação do TS. Um pacote de TS deve conter o cabeçalho obrigatório, e a ele devem suceder 184 bytes
de dados. Esses bytes restantes podem ser preenchidos somente com o \textit{Adaptation Field} ou com este
e dados de pacotes PES ou sessões de tabelas PSI. 

A norma ISO/IEC 13818 recomenda que a taxa de bits do TS deve ser constante, de modo que espaços vazios devem
ser preenchidos por \textit{stuffing bytes}. O \textit{Adaptation Field} pode também ser utilizado para
ocupar espaço vazio de um pacote do TS.

\begin{figure}
\centering
\includegraphics[width=0.6\linewidth]{figuras/TS_iso13818.png}
\caption{Esquema de formação de um pacote de TS.}
\label{fig:TS_iso13818}
\end{figure}

% ---

\subsection{PSI}

O PSI é o conjunto de informações especificas ao programa, ou \textit{Program Specific information}, em inglês,
e serve para organizar as informações de transmissão, sincronismo, decodificação e apresentação dos
fluxos de áudio, vídeo e dados. O PSI é organizado em forma de tabelas, cada uma com uma função específica no
padrão. As tabelas que servem a indicar ao decodificador onde estão no TS os fluxos de vídeo e áudio, chamadas
de PAT( Tabela de associação de programas) e PMT (Tabela de Mapeamento de Programas), são obrigatórias,
Há ainda um conjunto grande de tabelas que servem a outros fins, e que não são obrigatórias segundo a ISO13818,
mas que a ABNT NBR15603 define como obrigatórias.

A norma ISO define que as tabelas do PSI devem ser enviadas em seções e as seções podem ser segmentadas para
caberem nos pacotes de TS. As seções não podem ter mais de 1024 bytes, embora na prática o número de bytes 
para uma tabela obrigatória fique bem abaixo.

A sintaxe das tabelas é padronizada. Um campo no início da tabela (table\_ID) define qual o seu tipo
(PAT,PMT, etc.) e o campo section number serve como um contador caso a tabela seja dividida em diversas
seções. O campo version umos campos que seguem são personalizados version, section number, program number 

\section{Sincronismo}

O sincronismo é fundamental para o funcionamento de um sistema de transmissão de televisão, pois é
preciso garantir a reprodução do vídeo e do áudio simultaneamente e com a mesma referência temporal.
Para possibilitar tal sincronismo, um complexo sistema de sincronismo existe no padrão. O conceito
fundamental é a transmissão de um sinal de clock codificado nos pacotes do TS, o PCR, explicado a seguir.
Esse sinal é decodificado
e alimenta um sistema PLL no decodificador, para então ser utilizado nos fluxos de áudio e vídeo.
Os fluxos elementares, por sua vez, transportam referências temporais ao PCR nos campos PTS e DTS, explicados
a seguir, e assim são reproduzidos com base em uma mesma referência temporal.

\subsection{PCR}
O valor do PCR é uma espécie de amostragem do clock do encoder de vídeo ou áudio em um momento específico
da geração da unidade fundamental, seja um quadro de áudio ou vídeo, se o sistema trabalhar em tempo real.
Como no caso desde multiplexador não é, só é preciso definir
a frequência, que é definida pela norma em 27MHz, e calcular os valores sequenciais do PCR baseando-se 
na taxa de bits por segundo de cada um dos sinais de vídeo, ou áudio.

\begin{equation}
$$
PCR = F * ES_BR * 
$$
\end{equation}

O PCR é enviado em  um campo do cabeçalho do TS, e está dentro do adaptation field. Embora seja opcional enviá-lo,
nota-se pela leitura que é possível enviar o PCR em poucas repetições. Não é necessário enviá-lo junto
com todos os pacotes de TS. Pode-se enviar um pacote apenas com o PCR, utilizando um PCR\_ID, para resincronizar o
PLL do receptor a cada N segundos, N a definir.


\subsection{PTS / DTS}

Os campos Decoding Time Stamp (DTS) e Presentation Time Stamp (PTS) são outros dois referenciais de tempo,
derivados do PCR, e que servem a informar ao decodificador o instante de tempo em que devem ser decodificadas
e exibidas, respectivamente, as Access Units de vídeo e áudio. Uma Presentation Unit (PU) é definida pela
norma como sendo um quadro de áudio ou de vídeo, e uma Access Unit (AU) é a representação codificada 
da Presentation Unit.

A norma define claramente qual instante de tempo deve ser considerado para gerar o PTS ou o DTS:
\quotation{No caso do áudio, se um PTS está presente num cabeçalho de PES, este deve fazer referência à
primeira AU que comece no pacote. Uma AU de áudio começa em um pacote se o primeiro byte da AU de áudio
estiver presente no pacote. No caso do vídeo, se um PTS está presente no cabeçalho do pacote PES, este
deve se referir à AU que cujo primeiro \textit{start\_code} está neste pacote.
}

% ---

% ----------------------------------------------------------
% PARTE
% ----------------------------------------------------------
%\part{Resultados}
% ----------------------------------------------------------

% ---
% primeiro capitulo de Resultados
% ---
%\chapter{Resultados obtidos com o multiplexador}
% ---

% ---
%\section{Desempenho em velocidade de multiplexação}
% ---

%Para quantificar o tempo que o software desenvolvido leva para tratar fluxos de vídeo e áudio de tamanhos específicos, utilizou-se o comando \texttt{time}.

%\lipsum[21-22]

%% ---
%% segundo capitulo de Resultados
%% ---
%\chapter{Nam sed tellus sit amet lectus urna ullamcorper tristique interdum
%elementum}
%% ---
%
%% ---
%\section{Pellentesque sit amet pede ac sem eleifend consectetuer}
%% ---
%
%\lipsum[24]

% ----------------------------------------------------------
% Finaliza a parte no bookmark do PDF
% para que se inicie o bookmark na raiz
% e adiciona espaço de parte no Sumário
% ----------------------------------------------------------
\phantompart

% ---
% Conclusão (outro exemplo de capítulo sem numeração e presente no sumário)
% ---
\chapter[Conclusão]{Conclusão}
\addcontentsline{toc}{chapter}{Conclusão}
% ---

%\lipsum[31-33]
%Aqui serão escritas as conclusões do trabalho, com seriedade e racionalidade.


% ---
% Perspectivas (outro exemplo de capítulo sem numeração e presente no sumário)
% ---
%\chapter*[Perspectivas]{Perspectivas}
%\addcontentsline{toc}{chapter}{Perspectivas}
% ---

%\lipsum[31-33]
%Aqui serão escritas as perspectivas futuras para a continuação do desenvolvimento, com 
%comentários sobre o fato de que ainda haverá tempo para trabalhar no projeto até final
%de agosto.



% ----------------------------------------------------------
% ELEMENTOS PÓS-TEXTUAIS
% ----------------------------------------------------------
\postextual
% ----------------------------------------------------------

% ----------------------------------------------------------
% Referências bibliográficas
% ----------------------------------------------------------
\bibliography{abntex2-modelo-references}

% ----------------------------------------------------------
% Glossário
% ----------------------------------------------------------
%
% Consulte o manual da classe abntex2 para orientações sobre o glossário.
%
%\glossary

% ----------------------------------------------------------
% Apêndices
% ----------------------------------------------------------

% ---
% Inicia os apêndices
% ---
\begin{apendicesenv}

% Imprime uma página indicando o início dos apêndices
\partapendices

% ----------------------------------------------------------
\chapter{Tabelas selecionadas retiradas da norma ISO/IEC 13818-1}
% ----------------------------------------------------------

%\lipsum[50]

% ----------------------------------------------------------
\chapter{Trecho do código implementado em linguagem C}
% ----------------------------------------------------------
%\lipsum[55]

\end{apendicesenv}
% ---


% ----------------------------------------------------------
% Anexos
% ----------------------------------------------------------

% ---
% Inicia os anexos
% ---
\begin{anexosenv}

% Imprime uma página indicando o início dos anexos
\partanexos

% ---
%\chapter{Trechos extraídos da norma ISO/IEC 13818}
% ---
%\lipsum[30]

% ---
%\chapter{Cras non urna sed feugiat cum sociis natoque penatibus et magnis dis
%parturient montes nascetur ridiculus mus}
% ---

%\lipsum[31]

% ---
%\chapter{Fusce facilisis lacinia dui}
% ---

%\lipsum[32]

\end{anexosenv}

%---------------------------------------------------------------------
% INDICE REMISSIVO
%---------------------------------------------------------------------
\phantompart
\printindex
%---------------------------------------------------------------------

\end{document}
