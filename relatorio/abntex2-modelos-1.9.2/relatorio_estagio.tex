%% abtex2-modelo-trabalho-academico.tex, v-1.9.2 laurocesar
%% Copyright 2012-2014 by abnTeX2 group at http://abntex2.googlecode.com/ 
%%
%% This work may be distributed and/or modified under the
%% conditions of the LaTeX Project Public License, either version 1.3
%% of this license or (at your option) any later version.
%% The latest version of this license is in
%%   http://www.latex-project.org/lppl.txt
%% and version 1.3 or later is part of all distributions of LaTeX
%% version 2005/12/01 or later.
%%
%% This work has the LPPL maintenance status `maintained'.
%% 
%% The Current Maintainer of this work is the abnTeX2 team, led
%% by Lauro César Araujo. Further information are available on 
%% http://abntex2.googlecode.com/
%%
%% This work consists of the files abntex2-modelo-trabalho-academico.tex,
%% abntex2-modelo-include-comandos and abntex2-modelo-references.bib
%%

% ------------------------------------------------------------------------
% ------------------------------------------------------------------------
% abnTeX2: Modelo de Trabalho Academico (tese de doutorado, dissertacao de
% mestrado e trabalhos monograficos em geral) em conformidade com 
% ABNT NBR 14724:2011: Informacao e documentacao - Trabalhos academicos -
% Apresentacao
% ------------------------------------------------------------------------
% ------------------------------------------------------------------------

\documentclass[
	% -- opções da classe memoir --
	12pt,				% tamanho da fonte
%	openright,			% capítulos começam em pág ímpar (insere página vazia caso preciso)
	oneside,			% para impressão em verso e anverso. Oposto a oneside
	a4paper,			% tamanho do papel. 
	% -- opções da classe abntex2 --
	%chapter=TITLE,		% títulos de capítulos convertidos em letras maiúsculas
	%section=TITLE,		% títulos de seções convertidos em letras maiúsculas
	%subsection=TITLE,	% títulos de subseções convertidos em letras maiúsculas
	%subsubsection=TITLE,% títulos de subsubseções convertidos em letras maiúsculas
	% -- opções do pacote babel --
	%english,			% idioma adicional para hifenização
	brazil
	%french,				% idioma adicional para hifenização
	%spanish,			% idioma adicional para hifenização
	%brazil				% o último idioma é o principal do documento
	%english
	]{abntex2}

% ---
% Pacotes básicos 
% ---
\usepackage{lmodern}			% Usa a fonte Latin Modern			
\usepackage[T1]{fontenc}		% Selecao de codigos de fonte.
\usepackage[utf8]{inputenc}		% Codificacao do documento (conversão automática dos acentos)
\usepackage{lastpage}			% Usado pela Ficha catalográfica
\usepackage{indentfirst}		% Indenta o primeiro parágrafo de cada seção.
\usepackage{color}				% Controle das cores
\usepackage{graphicx}			% Inclusão de gráficos
\usepackage{microtype} 			% para melhorias de justificação
\usepackage{listings} 			% para llistas de codigo
\usepackage{caption} 			% para legendas nas subpictures
\usepackage{subcaption} 			% para legendas nas subpictures
% ---
		
% ---
% Pacotes adicionais, usados apenas no âmbito do Modelo Canônico do abnteX2
% ---
\usepackage{lipsum}				% para geração de dummy text
% ---

% ---
% Pacotes de citações
% ---
\usepackage[brazilian,hyperpageref]{backref}	 % Paginas com as citações na bibl
\usepackage[alf]{abntex2cite}	% Citações padrão ABNT

% --- 
% CONFIGURAÇÕES DE PACOTES
% --- 

% ---
% Configurações do pacote backref
% Usado sem a opção hyperpageref de backref
\renewcommand{\backrefpagesname}{Cited in page(s):~}
% Texto padrão antes do número das páginas
\renewcommand{\backref}{}
% Define os textos da citação
\renewcommand*{\backrefalt}[4]{
	\ifcase #1 %
		No citation in the text.%
	\or
		Cited in page #2.%
	\else
		Cited #1 times in pages #2.%
	\fi}%
% ---

% ---
% Informações de dados para CAPA e FOLHA DE ROSTO
% ---
\titulo{Estágio Supervisionado na RBS TV em 2014/1 \\ \- \\ Projeto e implantação de televisão digital \\ em dez cidades do RS e SC}
\autor{Lucas Pereira Endres}
\local{Porto Alegre}
\data{2014}
%\orientador{Altamiro Amadeu Susin}
%\coorientador{André Borin Soares}
\instituicao{%
  Universidade Federal do Rio Grande do Sul
  \par
  Escola de Engenharia
  \par
  Departamento de Engenharia Elétrica}
%\tipotrabalho{Graduation Thesis}
\tipotrabalho{Relatório de Estágio Supervisionado}
% O preambulo deve conter o tipo do trabalho, o objetivo, 
% o nome da instituição e a área de concentração 
\preambulo{}
% ---


% ---
% Configurações de aparência do PDF final

% alterando o aspecto da cor azul
\definecolor{blue}{RGB}{41,5,195}

% informações do PDF
\makeatletter
\hypersetup{
     	%pagebackref=true,
		pdftitle={\@title}, 
		pdfauthor={\@author},
    	pdfsubject={\imprimirpreambulo},
	    pdfcreator={LaTeX with abnTeX2},
		pdfkeywords={Multiplexation}{MPEG2}{ISDB-T}{Transport Stream}, 
		colorlinks=true,       		% false: boxed links; true: colored links
    	linkcolor=blue,          	% color of internal links
    	citecolor=blue,        		% color of links to bibliography
    	filecolor=magenta,      		% color of file links
		urlcolor=blue,
		bookmarksdepth=4
}
\makeatother
% --- 

% --- 
% Espaçamentos entre linhas e parágrafos 
% --- 

% O tamanho do parágrafo é dado por:
\setlength{\parindent}{1.3cm}

% Controle do espaçamento entre um parágrafo e outro:
\setlength{\parskip}{0.2cm}  % tente também \onelineskip

% ---
% compila o indice
% ---
\makeindex
% ---

\lstset{
	basicstyle=\footnotesize\ttfamily,
	%framextopmargin=50pt,
	%frame=lrtb
    language=C,
    frame=single,
    tabsize=2,
    showspaces=false,
    showstringspaces=false,
    keywordstyle=\color{blue},
    %morekeywords={QStringList,QDate,QString,QIODevice},
    commentstyle=\color{CadetBlue},
    %caption={Zistenie, či sme v daný deň, už záznam o rýchlosti uložili},
    breaklines=true
	}

% ----
% Início do documento
% ----
\begin{document}

% Retira espaço extra obsoleto entre as frases.
\frenchspacing 

% ----------------------------------------------------------
% ELEMENTOS PRÉ-TEXTUAIS
% ----------------------------------------------------------
% \pretextual

% ---
% Capa
% ---
\imprimircapa
% ---

% ---
% Folha de rosto
% (o * indica que haverá a ficha bibliográfica)
% ---
\imprimirfolhaderosto*
% ---

% ---
% inserir o sumario
% ---
\pdfbookmark[0]{\contentsname}{toc}
\tableofcontents*
\cleardoublepage
% ---

% ----------------------------------------------------------
% ELEMENTOS TEXTUAIS
% ----------------------------------------------------------
\textual

% ----------------------------------------------------------
% Introdução (exemplo de capítulo sem numeração, mas presente no Sumário)
% ----------------------------------------------------------
\chapter[Introdução]{Introdução}
%\addcontentsline{toc}{chapter}{Introduction}
% ----------------------------------------------------------

Historicamente, a televisão está presente nos lares da maioria dos cidadãos brasileiros e é a principal fonte de entretenimento e informação à população. Ao longo dos últimos 10 anos, novas mídias digitais, como computadores e telefones celulares estão sendo adotadas pela população de diferentes classes sociais. Um recente relatório \cite{pnad2011} diz que em 2011, 69\% da população tinha uma linha de telefone celular. Apesar de significativa, a participação deste meio ainda é muito abaixo da televisão, cuja área de cobertura atinge 100\% do território via satélite \cite{StarOne}, e em torno de 98\% da população por terra \cite{globo}. A televisão é, portanto, o principal canal de comunicação disponível para o público em geral no Brasil.

%No Brasil, desde 1994, o governo e empresas privadas financiaram testes e pesquisas técnicas para comparar o desempenho de três sistemas de televisão digital que eram conhecidos por serem eficientes em seus países de origem na época: ATSC \cite{ATSC}, desenvolvido nos Estados Unidos; DVB-T \cite{DVB}, desenvolvido por um consórcio de empresas para uso em países europeus; e ISDB, desenvolvido no Japão pela ARIB \cite{ARIB}. Os três sistemas têm semelhanças e diferenças em relação a codificação de vídeo e áudio: por exemplo, tanto o DVB e ISDB utilizam a infra-estrutura de transporte do padrão H.222, mas diferem nos esquemas de modulação. Após as avaliações, foi determinado que o sistema com o melhor desempenho para o território brasileiro seria baseado no ISDB japonês, com algumas modificações, como apontado em \cite{decreto8061}.

Um sistema de televisão digital é composto basicamente pelo grupo de equipamentos apresentado em \autoref{fig:diagrama_blocos_tvd}. No início do fluxo de sinal, existem os elementos captura de vídeo e áudio, tais como câmaras e microfones, e o sinal bruto pode ser analógico ou digital, dependendo da tecnologia de captura utilizada. Uma vez capturado, o vídeo e o áudio são compactados e codificados por meio de encoders correspondentes. A saída dos codificadores são bitstreams padronizados chamados Elementary Streams. Uma vez que os fluxos elementares saem dos codificadores, eles entram no multiplexador, onde todos os fluxos são empacotados e embutidos em um único fluxo. O que se segue é um processo essencial para a robustez do sistema: o fluxo é protegido pelo uso de códigos de correção de erros para resistir ao ruidoso canal multipercurso entre a emissora e o receptor. Finalmente, uma modulação digital é aplicada e o fluxo modulado é enviado para a antena.
 
 \begin{figure}[!h]
\centering
\caption{Block diagram showing digital television basic signal flow.}
\includegraphics[width=1\linewidth]{figuras/diagrama_blocos_tvd.png}
\label{fig:diagrama_blocos_tvd}
\end{figure}
 
O Decreto presidencial número 8061, de 29 de julho de 2013, estabelece o cronograma de desligamento dos sistemas de transmissão de televisão analógica. Até 31 de dezembro de 2018, todos os transmissores analógicos devem ser desativados e os canais devem ser liberados. As outorgas serão revogadas e os canais devolvidos à união, que planeja utilizar a faixa dos 700MHz para implantar a tecnologia LTE, de telefonia móvel 4G, nesta faixa do espectro.

É portanto de grande interesse das emissoras de televisão atualizar seus transmissores para a tecnologia digital, para a manutenção da audiência. Se por um lado a nova tecnologia entrega melhor qualidade de vídeo e áudio ao telespectador, por outro lado a transmissão analógica deverá acabar e quem não se atualizar deverá cessar as transmissões. A televisão aberta digital representa para as emissoras uma grande oportunidade de concorrência com as operadoras de televisão paga, uma vez que a TV paga já é digitalizada há anos e sem essa oportunidade de transmissão, as emissoras perdiam a audiência dos usuários de televisores de alta definição para a TV paga.

Baseado no diagrama de blocos apresentado na \autoref{fig:diagrama_blocos_tvd}, é possível perceber que não basta atualizar apenas parte da cadeia de transmissão para a tecnologia digital a fim de entregar um produto de melhor qualidade ao telespectador. É necessário que os equipamentos de aquisição e manipulação sejam também atualizados para obter ganho efetivo de qualidade de imagem e áudio na recepção. Assim, é preciso considerar que a atualização de uma emissora de televisão requer um investimento considerável no projeto e aquisição dos equipamentos, bem como na implantação dos equipamentos.

Neste contexto de atualização do sistema de transmissão, a equipe de engenharia de projetos e implantação da RBS TV projetou e instalou as emissoras de televisão digital de 10 cidades no interior do Rio Grande do Sul e Santa Catarina durante o período de janeiro de 2013 até maio de 2014.

% ----------------------------------------------------------
% PARTE
% ----------------------------------------------------------
%\part{Preparação da pesquisa}
% ----------------------------------------------------------

% ---
% Capitulo com exemplos de comandos inseridos de arquivo externo 
% ---
%\include{abntex2-modelo-include-comandos}
% ---

% ----------------------------------------------------------
% PARTE
% ----------------------------------------------------------
%\part{Referenciais teóricos}
% ----------------------------------------------------------

% ---
% Capitulo de revisão de literatura
% ---
\section{Apresentação da Empresa e do Estágio}

A RBS TV é uma emissora de televisão afiliada da Rede Globo, com atuação nos estados do Rio Grande do Sul e Santa Catarina, na região sul do Brasil. Além de retransmitir a programação da Rede Globo na maior parte do tempo, também exibe produções próprias diariamente, nos segmentos informativos e de entretenimento.

Para garantir a ampliação e a manutenção de sua infraestrutura, a empresa tem uma grande equipe técnica que atua projetando a implantação de novas tecnologias e prestando suporte à infraestrutura existente. O setor técnico é composto de engenheiros e técnicos formados em assuntos associados a televisão, como Engenharia Elétrica, Engenharia de Telecomunicações, Engenharia de Computação, Técnico em Eletrônica e Técnico em Telecomunicação.

A equipe da engenharia de projetos tem como objetivo projetar e implantar soluções que agreguem as inovações tecnológicas atuais ao fluxo de trabalho das equipes de jornalismo e produção.

\section{Apresentação do Projeto}

O projeto específico em que o estágio ocorreu consiste em projetar e instalar geradoras de televisão em dez cidades com mais de 50 mil habitantes do interior do Rio Grande do Sul e Santa Catarina. São elas Santa Cruz do Sul, Bagé, Uruguaiana, Rio Grande, Erechim e Cruz Alta e Santa Rosa no Rio Grande do Sul, e Criciúma, Chapecó e Joaçaba, em Santa Catarina.

São tarefas da equipe o projeto técnico e a escolha dos equipamentos utilizados nas instalações. Optou-se por criar um padrão de emissora que foi replicado para todas as dez geradoras, visando minimizar o retrabalho de projeto e facilitar a atuação da equipe de suporte, uma vez que com um projeto padrão os equipamentos são iguais em todas as sedes.

A equipe é dividida em duas áreas: gestão e implantação. A área de gestão é responsável pela interface com outros setores internos da empresa, como comercial, suprimentos, logística, financeiro; bem como fornecedores externos de produtos e serviços. A área de implantação é responsável pelo projeto técnico, bem como por visitar as emissoras, executar a instalação dos equipamentos e supervisionar a atuação dos terceiros que atuam lá.

\section{Descrição das tarefas executadas}

As atividades do estágio foram desenvolvidas na sede da empresa, em Porto Alegre, vinculadas tanto à equipe de gestão quanto à equipe de implantação. O estagiário constituiu a interface entre as duas áreas, dando suporte à equipe que estava em constantes visitas ao interior para assegurar que o projeto seria executado corretamente.

As principais tarefas exercidas durante o estágio foram:

\begin{itemize}

\item parametrização e configuração dos equipamentos, a fim de respeitar as normas brasileiras e o padrão de qualidade de transmissão objetivado pela empresa;

\item coordenação do envio e recebimento dos equipamentos para as dez geradoras, programando datas de coletas e entregas dos equipamentos de Porto Alegre para as geradoras;

\item elaboração de documentos técnicos tutoriais para instruir os técnicos locais e os operadores sobre o funcionamento dos equipamentos;

\item elaboração de diagramas unifilares do fluxo de sinal de vídeo, áudio e dados ao longo da cadeia de transmissão

\item elaboração de modelos em três dimensões dos ambientes envolvidos nas geradoras, a fim de prever possíveis limitações de espaço e visando otimizar o espaço dos ambientes, prezando pela ergonomia dos operadores;

\item avaliação técnica de soluções disponíveis no mercado para problemas que ainda não tinham sido solucionados pela equipe, como o isolamento e a monitoração da conexão com terceirizados;
\end{itemize}

\chapter{Desenvolvimento}

\section{Diagramas de fluxo de sinal}

A documentação é fundamental para a manutenção de qualquer infraestrutura. Sem um diagrama atualizado, é praticamente impossível manter o sistema funcionando corretamente sem depender de alguém que conheça todo o sistema de memória. Assim, foram confeccionados diagramas unifilares em Microsoft Visio de cada parte do sistema. Os diagramas são confidenciais e não puderam ser replicados no relatório.

\section{Modelos em três dimensões}

Os desenhos em três dimensões dos ambientes técnicos e de operações possibilitaram prever os caminhos por onde passariam os cabos de alimentação elétrica, os cabos de vídeo e de áudio, bem como os cabos de rede e de controle, e também as eletrocalhas utilizadas para conduzir os cabos. Também assim, previu-se as reformas civis, de construção e demolição de paredes para reorganizar os ambientes.

Com estes desenhos preliminares, foi possível estimar o consumo de cabos para cada instalação e também prever caminhos críticos por onde haveria dificuldade de transpor determinados obstáculos, como vigas que deveriam ser contornadas ou sobreposição de calhas na chegada aos racks de equipamentos. Os desenhos visaram otimizar os aspectos ergonômicos de trabalho e a funcionalidade dos ambientes, economizar materiais e minimizar o retrabalho futuro.

O software escolhido para realizar os desenhos é o Google Sketchup. A seguir é apresentado um exemplo de utilização do software para prever o caminho de uma eletrocalha desviando de uma viga. Na \autoref{fig:real_sketchup_desenho} vê-se o desenho previamente realizado e na \autoref{fig:real_sketchup} vê-se o que de fato foi executado.

\begin{figure}[!h]
\centering
\caption{Desenho do projeto em 3D do ambiente.}
\includegraphics[width=0.8\linewidth]{figuras/real_sketchup_desenho.jpg}
\\Fonte: O autor.
\label{fig:real_sketchup_desenho}
\end{figure}

\begin{figure}[!h]
\centering
\caption{Foto real do ambiente projetado.}
\includegraphics[width=0.8\linewidth]{figuras/real_sketchup.jpg}
\\Fonte: Reprodução RBSTV.
\label{fig:real_sketchup}
\end{figure}

\section{Descritivo de Equipamentos}

Na instalação padrão de uma emissora de televisão são necessários uma centena de equipamentos para exibição do conteúdo, monitoração, seleção das fontes de sinal, comutação automática em caso de falha e controle dos periféricos. Seria inviável descrever todos os equipamentos neste relatório suscinto, então optou-se por abordar aspectos cruciais para o funcionamento em alguns breves parágrafos.

\subsection{Sinais de Referência}

Para garantir o correto funcionamento dos equipamentos em conjunto é indispensável utilizar um sinal que garanta o sincronismo entre todos eles. Este sinal funciona como se fosse um sinal de \textit{clock} global do sistema. O sinal utilizado na instalação é um sinal de Black Burst de vídeo analógico, que é basicamente um pulso de sinal com taxa de repetição constante e forma de onda conhecida, de modo que é ideal para servir de sinal de sincronismo.

O equipamento responsável por gerar este sinal é o Pulse Generator PG8000 da Tektronix. Além do sinal de Black Burst para sincronismo, ele também pode gerar sinais de vídeo em padrões analógicos e digitais, para testes de transmissão, como tons de áudio de frequências definidas ou quadros de vido estáticos com barras coloridas.

\subsection{Redundância}

A redundância na geração dos sinais é indispensável para se garantir um tempo médio entre falhas baixo. A RBS TV assume com a Rede Globo contratos que exigem um tempo máximo de falhas, e cada falha deve ser reportada à emissora.

Nessas condições, optou-se por projetar um sistema com dois caminhos independentes de sinal, desde as fontes de vídeo e áudio até os transmissores. O objetivo é prevenir-se de eventuais falhas nos equipamentos e minimizar o tempo fora de serviço.

Um dos equipamentos responsáveis é o Automatic ChangeOver ACO6800+ISCST da Imagine Communications\cite{harris}, cujo diagrama de blocos interno pode ser visto na \autoref{fig:aco_block_diag}.

\begin{figure}[!h]
\centering
\caption{Diagrama de blocos do ACO6800+ISCST.}
\includegraphics[width=0.5\linewidth]{figuras/aco_block_diag.png}
\\Fonte: O autor.
\label{fig:aco_block_diag}
\end{figure}

Vê-se na figura a presença de dois relés mecânicos, responsáveis por, em caso de pane no dispositivo, efetuar o \textit{bypass} do sinal de maneira física. Quando estiver em operação normal, o ACO envia o sinal da entrada In1 para as saídas Out1 e Out2, enquanto monitora o sinal de vídeo da entrada In1. Em caso de falha no sinal, ele comuta automaticamente as saídas para a entrada In2.

O equipamento conta com uma série de configurações possíveis para operação automática: é possível, por exemplo escolher os tipos de falha de vídeo que devem ser usados como gatilho para a comutação, bem como o tempo que se deve esperar para comutar. Ainda pode-se optar por fazer uma comutação manual utilizando um painel acessório através do conector GPI(General Purpose Interface).

É interessante notar, ainda, que o equipamento dispõe de uma entrada de sinal de referência, ou \textit{Genlock}, pois é preciso que a comutação seja feita de maneira sincronizada com a mudança de um quadro de vídeo para o próximo, para evitar que haja variações na imagem.

\section{Medições de qualidade de sinal digital}

O diagrama de olho permite avaliar a qualidade do sinal analógico, da camada física, que transporta o sinal digital. O jitter é uma medida da variação do atraso relativo entre os sinais de vídeo, causado basicamente por problemas com cabos de má qualidade, muito extensos ou com falhas, como dobras acentuadas ou rompimento da malha.

O ruído é causado pela indução magnética dos equipamentos ao redor do cabo. Caso ele tenha uma malha de proteção fina, pode deixar o condutor de sinal exposto a induções de sinais externos ao que se deseja transportar, originando ruido.

A seguir são apresentadas duas imagens com diagramas de olho. Na \autoref{fig:olho_pouco_jitter} há pouco jitter e pouco ruido e na \autoref{fig:olho_muito_jitter} há consideravelmente bastante jitter e ruido. Um nível alto de jitter pode levar o receptor a não diferenciar os instantes de mudança de um valor binário para o outro, enquanto que o ruído interfere nos níveis de sinal para cada valor lógico.

\begin{figure}[!h]
\centering
\caption{Diagrama de olho com pouco jitter e ruido.}
\includegraphics[width=0.5\linewidth]{figuras/olho_pouco_jitter.png}
\\Fonte: Tektronix.
\label{fig:olho_pouco_jitter}
\end{figure}

\begin{figure}[!h]
\centering
\caption{Diagrama de olho com muito jitter e ruido.}
\includegraphics[width=0.5\linewidth]{figuras/olho_muito_jitter.png}
\\Fonte: Tektronix.
\label{fig:olho_muito_jitter}
\end{figure}

% ---
% Conclusão (outro exemplo de capítulo sem numeração e presente no sumário)
% ---
\chapter[Conclusões]{Conclusões}
%\addcontentsline{toc}{chapter}{Conclusions and Future Development}
% ---

O local de trabalho é muito bom em termos de aprendizado técnico e prático dos afazeres de uma equipe de projetos. Com experiência em gestão de projetos, o gerente foi responsável e buscou realizar reuniões diárias nos primeiros meses de estágio para garantir a ambientação do estagiário na equipe.

O objetivo de desenvolvimento pessoal foi atingido satisfatoriamente, com aquisição de capacidades técnicas específicas que não se tem a oportunidade de desenvolver na UFRGS por falta de disciplinas específicas sobre televisão analógica e digital. Foi possível aprender como é o fluxo de sinal desde a captação, passando pelo armazenamento em diferentes mídias, depois sendo codificado e sendo finalmente transmitido nos padrões analógico e digital.

Foi possível também compreender as condições políticas e contratuais que interferem nos aspectos técnicos e de planejamento, como a necessidade de redundância de toda a cadeia de transmissão para minimizar os tempos de falha.

É também clara a utilidade do planejamento do layout dos ambientes para prever dificuldades e evitar retrabalhos após a instalação, seja pela possibilidade de identificar obstáculos físicos, seja por prever a posição específica dos operadores considerando aspectos ergonômicos da posição de trabalho. Com isso, pode-se até evitar eventuais lesões dos operadores.

A equipe é preocupada em integrar o estagiário nos diferentes núcleos envolvidos na gestão do projeto. É possível ao estagiário interagir com as equipes de suprimentos, logística e financeira, além de participar ativamente do projeto técnico evidentemente. Com isso, ainda que o estagiário não tenha conhecimentos específicos logo que ingressa na vaga, ele pode participar de outros procedimentos inerentes à gestão de um projeto, mas que não se aprende nas disciplinas teóricas do curso.

% ----------------------------------------------------------
% ELEMENTOS PÓS-TEXTUAIS
% ----------------------------------------------------------
\postextual
% ----------------------------------------------------------

% ----------------------------------------------------------
% Referências bibliográficas
% ----------------------------------------------------------
\bibliography{relatorio_estagio.bib}

\end{document}
